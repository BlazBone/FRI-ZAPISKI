\documentclass{article}
\usepackage[margin=0.15cm]{geometry}
\usepackage{amsmath}
\usepackage{multicol}

\setlength{\columnseprule}{0.5pt}

\begin{document}

\begin{center}
    {\small TOC/FRI \par}
\end{center}

\begin{multicols}{2}

\section{\underline{Finite Automata}}

\textbf{1.1 Deterministic finate automaton - DFA}\\
is a 5-tuple $(Q, \Sigma, \delta, q_0, F)$ where:
\begin{itemize}
    \setlength\itemsep{-0.4em}
    \item $Q$ is a finate set of \textit{states},
    \item $\Sigma$ is a finite \textit{input alphabet}
    \item $q_0 \in Q$ is the \textit{initial state},
    \item $F \subseteq Q$ is the set of \textit{final states}, and
    \item $\delta$ is the \textit{transition function}, i.e. $\delta: Q \times \Sigma \rightarrow Q$ \\
    For each state, there must be a transition for every input symbol out of $\Sigma$.
\end{itemize}

\textbf{exp.} Dfa for finding modulo of binary numbers\\
Suppose our modulo is $m$. Then for every possible remainder, there must be
a state in fa $\{q_0, q_1, \dots, q_{m-1}\}$.
\begin{itemize}
    \setlength\itemsep{-0.4em}
    \item state $q_0$ : $m*k + 0$ \\
        $m*k\; |0 \Rightarrow$  $2*(5k) + 0 = m*k + 0 $ (on $0$, we go to $q_0$) \\
        $m*k\; |1 \Rightarrow$  $2*(5k) + 1 = m*k + 1 $ (on $1$, we go to $q_1$)
    \item state $q_1$ : $k + 1$ \\
        $m*k\; |0 \Rightarrow$  $2*1 + 0 = 2 $ (go to $q_2$)\\ 
        $m*k\; |1 \Rightarrow$  $2*1 + 1 = 3 $ (go to $q_3$)
    \item state $q_{m-1}$ : $k + (m-1)$ \\
        $m*k\; |0 \Rightarrow$  $2*(m-1) + 0$ \\ 
        $m*k\; |1 \Rightarrow$  $2*(m-1) + 1$
\end{itemize}
If your remainder is bigger than $m$, then you must modulo it!

\textbf{1.2 Nondeterministic finate automaton - NFA}\\
is a 5-tuple $(Q, \Sigma, \delta, q_0, F)$ where:
\begin{itemize}
    \setlength\itemsep{-0.4em}
    \item $Q, \Sigma, q_0, F$ \textit{read dfa}
    \item $\delta$ is the \textit{transition function}, i.e. $\delta: Q \times \Sigma \rightarrow 2^Q$ \\
        That is $\delta(q, a)$ is the \textit{set} of all states $p$ such that there is a transition labeled from $a$ to $p$.
\end{itemize}

\textbf{1.3 NFA} with epsilon moves - $NFA_{\epsilon}$\\
is a 5-tuple $(Q, \Sigma, \delta, q_0, F)$ where:
\begin{itemize}
    \setlength\itemsep{-0.4em}
    \item $Q, \Sigma, q_0, F$ \textit{read dfa}
    \item $\delta$ is the \textit{transition function}, i.e. $\delta: Q \times (\Sigma \cup \{\epsilon\} )\rightarrow 2^Q$ \\
    That is $\delta(q, a)$ is the \textit{set} of all states $p$ such that there is a transition labeled from $a$ to $p$, \textit{where \textbf{a} is either a symbol in} $\Sigma$  \textit{or} $\epsilon$.
\end{itemize}
$\epsilon$-closure defines which $\epsilon$ transitions are allowed from a single state in a fa (set of states we can reach).

\textbf{exp.} NFA for $L^{c}$\\
\begin{center}
    \begin{math}
        NFA(L) \rightarrow DFA(L) \rightarrow DFA(L^{c})
    \end{math}
\end{center}
Due to the properties of \textbf{DFA}, the complementation is applied just by switching final and non-final states of fa.

\section{\underline{Regular expressions}}


\textbf{2.1 Regular operations}\\
Let $L_1, L_2$ be some regular languages. Then their
\begin{itemize}
    \setlength\itemsep{-0.4em}
    \item \textbf{union} $\rightarrow L_1 \cup L_2 = \{\forall x : x \in L_1\; or\; x \in L_2 \} $
    \item \textbf{concatenation} $\rightarrow L_1 . L_2 = L_1L_2$
    \item \textbf{kleene closure} $\rightarrow = L^*$
    \item \textbf{interscetion} $\rightarrow L_1 \cap L_2$
    \item \textbf{complementation} $\rightarrow \overline{L}$
\end{itemize}
are also regular languages. Regexp are equvalent with NFA.

\textbf{2.2 Pumping lemma for regular languages}
Let $R$ be a class of regular languages.
Then language $L \in R \rightarrow \exists n > 0:$
\begin{center}
    \begin{math}
        \forall z \in L, |z| \geq n:
    \end{math}
    \begin{math}
        \exists u,v,w:\; |uv| \leq n, |v| \geq 1, z = uvw \rightarrow \forall i \geq 0: uv^iw \in L
    \end{math}
\end{center}
\textit{if we negate lemma, we can prove that some languages are irregular}
\begin{center}
    \begin{math}
        \forall n > 0:  \exists z \in L, |z| \geq n
    \end{math}
    \begin{math}
        \forall u,v,w:\; |uv| \leq n, |v| \geq 1, z = uvw \rightarrow \exists i \geq 0: uv^iw \notin L \Rightarrow L \notin R
    \end{math}
\end{center}

\section{\underline{Context-free grammars}}

\textbf{3.1 Definition:} A context-free grammar (CFG) \\
is a 4-tuple $G = (V, T, P, S)$ where:
\begin{itemize}
    \setlength\itemsep{-0.4em}
    \item $V$ is a finite set of \textit{variables}
    \item $T$ is a finite set of \textit{terminals}
    \item $P$ is a finite set of \textit{productions}\\
        each of which is of the form $A \rightarrow \alpha$,\\
        where $A \in V$ and $\alpha$ is a word in the language $(V \cup T)^*$
    \item $S$ is a special variable called the \textit{start symbol}
\end{itemize}
\textbf{Ambiguity}:
A CFG is said to be ambiguous if some word has more than one derivation tree.

\textbf{exp.} regex to CFG conversion\\
Suppose we have a regex: $a(ab)^*bb(aa+b)^*a$\\
Then we could model a CFG for it as:
\begin{itemize}
    \setlength\itemsep{-0.5em}
    \item $S \rightarrow XYZUV$
    \item $X \rightarrow a$
    \item $Y \rightarrow abY | \epsilon$
    \item $Z \rightarrow bb$
    \item $U \rightarrow aaU|bU|\epsilon$
    \item $V \rightarrow a$
\end{itemize}

\textbf{3.2 Pumping lemma for context-free languages}
Let $L$ be a CFL.
\begin{center}
    \begin{math}
        \exists n > 0:
    \end{math}\\
    \begin{math}
        \forall z \in L, |z| \geq n:
    \end{math}\\
    \begin{math}
        \exists u,v,w,x,y:\; |vwx| \leq n, |vx| \geq 1
    \end{math}\\
    \begin{math}
        z = uvwxy \rightarrow \forall i \geq 0: uv^iwx^iy \in L
    \end{math}
\end{center}
\textit{if we negate lemma, we can prove that some languages are not context-free.}
\begin{center}
    \begin{math}
        \forall n > 0:
    \end{math}\\
    \begin{math}
        \exists z \in L, |z| \geq n:
    \end{math}\\
    \begin{math}
        \forall u,v,w,x,y:\; |vwx| \leq n, |vx| \geq 1
    \end{math}\\
    \begin{math}
        z = uvwxy \rightarrow \exists i \geq 0: uv^iwx^iy \notin L
    \end{math}
\end{center}

\section{\underline{Pushdown Automata}}
\textbf{4.1 Definition:} A pushdown automaton (PDA) \\
is a 7 tuple $M =(Q, \Sigma, \Gamma, \delta, q_0, Z_0, F)$ where:
\begin{itemize}
    \setlength\itemsep{-0.4em}
    \item $Q, \Sigma, q_0, F$ read dfa
    \item $\Gamma$ is the \textit{stack alphabet}
    \item $Z_0 \in \Gamma$ is the \textit{start stack symbol}, and
    \item $\delta$ is the \textit{transition function}\\
        i.e. a mapping from $Q \times (\Sigma \cup \{\epsilon\}) \times \Gamma$ to finite subsets of $Q \times \Gamma^*$\\
        $\rightarrow 2^{Q \times \Gamma^*}$
\end{itemize}
\textbf{4.2} Accepted languages of the PDA\\
For PDA $M =(Q, \Sigma, \Gamma, \delta, q_0, Z_0, F)$  we define two languages:
\begin{itemize}
    \item $L(M)$, the \textit{language accepted by final state}, to be \\
        \begin{math}
            L(M) = 
                \{
                    w \in \Sigma^* | (q_0, w, Z_0) \rightarrow^* (p, \epsilon, \gamma)$\\$
                    for\; some\; p\; \in F\; and\; \gamma \in \Gamma^*
                \}
        \end{math}
    \item $L(M)$, the \textit{language accepted by empty stack}, to be \\
        \begin{math}
            N(M) = 
                \{
                    w \in \Sigma^* | (q_0, w, Z_0) \rightarrow^* (p, \epsilon, \epsilon);\
                    for\; some\; p\; \in Q
                \}
        \end{math}
\end{itemize}

\textbf{4.3} The class of CFLs is \textbf{closed} under:\\
union, concatenation, kleene closure, substitution, inverse homomorphism\\
The class of CFLs is \textit{not} \textbf{closed} under: interscetion, complementation.\\
But is closed for intersection if both CFL represent some regular sets.

\section{\underline{Touring Machines}}

\textbf{5.1 Definition:} A basic Toruing Machine (TM)\\
is a 7-tuple $M = \{Q, \Sigma, \Gamma, \delta, q_0, B, F\}$ where:
\begin{itemize}
    \setlength\itemsep{-0.4em}
    \item $Q$ is a finite set of \textit{states}
    \item $\Sigma$ is the \textit{input alphabet} 
    \item $\Gamma$ is the \textit{tape alphabet} \small{$B \in \Gamma => \Sigma \subseteq \Gamma$}
    \item $\delta$ is the \textit{transition function}
    \item $q_0$ is the \textit{initial state} and,
    \item $F \subseteq Q$: is the set of \textit{final states}
\end{itemize}

\textbf{5.2 TM modifications:}\\
\begin{itemize}
    \setlength\itemsep{-0.4em}
    \item Finite storage $\Rightarrow \delta: Q \times \Gamma \times \Gamma^k \rightarrow Q \times \Gamma \times \{L, R, S\} \times \Gamma^k$
    \item Multiple track tape $\Rightarrow \delta: Q \times  \Gamma^{tk} \rightarrow Q \times \Gamma^{tk} \times \{L, R, S\}$
    \item Two-way infinite tape $\Rightarrow \delta: Q \times  \Gamma \rightarrow Q \times \Gamma \times \{L, R, S\}$
    \item Multiple tapes $\Rightarrow \delta: Q \times  \Gamma^{tp} \rightarrow Q \times (\Gamma \times \{L, R, S\})^{tp}$
    \item Multidimensional tape $\Rightarrow \delta: Q \times  \Gamma \rightarrow Q \times \Gamma \times \{L_1, R_1, \dots, L_d, R_d, S\}$
\end{itemize}

\textbf{5.3 Universal Touring Machine} (UTM)\\
is a TM, that accepts some \textit{Touring machine} \textbf{M} description and a \textit{word} \textbf{w}. 
The universal TM then decides if $w \in L(M)$.\smallskip

[TM description$|$w]
$111<q_1>11<q_2>11\dots11<q_k>111w$ \smallskip

$\rightarrow$ language $L$ is \textbf{semi-decidable}, if there exists a TM, which:\\
    for every $w \in L$, TM halts in a final state

$\rightarrow$ language $L$ is \textbf{decidable}, if there exists a TM, which:\\
    for every $w \in L$, TM halts in a final state\\
    for every $w \notin L$, TM halts in a non-final state


$\rightarrow$ language $L$ is \textbf{undecidable}, if it is not decidable.

\textbf{5.4 Theorems for sets}\\
Let $S$, $A$, $B$ be arbitrary sets. Then:
\begin{itemize}
    \setlength\itemsep{-0.4em}
    \item $S$ is \textit{decidable} $\Rightarrow$ $S$ is \textit{semi-decidable}
    \item $S$ is \textit{decidable} $\Rightarrow$ $\overline{S}$ is \textit{decidable}
    \item $S$ and $\overline{S}$ are \textit{semi-decidable} $\Rightarrow$ $S$ is \textit{decidable}
    \item $A$ and $B$ are \textit{semi-decidable} $\Rightarrow$ $A \cap B $ $\&$ $A \cup B$ are \textit{semi-decidable}
    \item $A$ and $B$ are \textit{decidable} $\Rightarrow$ $A \cap B $ $\&$ $A \cup B$ are \textit{decidable}
\end{itemize}

\textbf{5.5 Three possibilities for set complementation:}
\begin{itemize}
    \setlength\itemsep{-0.2em}
    \item $S$ and $\overline{S}$ are \textit{decidable}
    \item $S$ and $\overline{S}$ are \textit{undecidable}, one is semi, and the other is not.
    \item $S$ and $\overline{S}$ are \textit{undecidable}, and neither is semi-decidable.
\end{itemize}

\textbf{5.6 Known languages:}\\
\begin{itemize}
    \setlength\itemsep{-0.2em}
    \item \textit{Diagonalizable language} $\rightarrow L_d = \{<M >| <M> \notin L(M)\}$ - undecidable / not semi-decidable.
    \item \textit{Universal language} $\rightarrow L_u = \{(<M>, w) | w \in L(M)\}$ - semi-decidable, but not decidable.
    \item \textit{Empty language} $\rightarrow L_e \{<M> | L(M) = \{\}\}$ - undecidable
    \item \textit{Non-Empty language} $\rightarrow L_{ne} = \{<M> | L(M) \neq \{\}\}$ - semi-decidable, but not decidable.
\end{itemize}

\textbf{5.6 Rice's theorem for (not)semi-decidabilty:}
\begin{enumerate}
    \setlength\itemsep{-0.4em}
    \item $L \in S \wedge L \subseteq L' \Rightarrow L' \in S$
    \item $L \in S \wedge L\; infinite \Rightarrow \exists L' \subseteq L: L \in S,\; L'\; finite$
    \item innumerability of final sets in $S$
\end{enumerate}
\begin{center}
    \begin{math}
        (1) \wedge (2) \wedge (3) \Leftrightarrow L_s $is semi-decidable$
    \end{math}
\end{center}


\section{\underline{Complexity classes}}

\textbf{6.1 In terms of formal languages:}
\begin{itemize}
    \setlength\itemsep{-0.4em}
    \item DTIME($T(n)$)  = $\{ L |$ L is a language $\wedge$ L has time complexity T(n)$\}$
    \item DSPACE($S(n)$) = $\{ L |$ L is a language $\wedge$ L has space complexity S(n)$\}$
    \item NTIME($T(n)$)  = $\{ L |$ L is a language $\wedge$ L has nondet. time complexity T(n)$\}$
    \item DSPACE($S(n)$) = $\{ L |$ L is a language $\wedge$ L has nondet. space complexity S(n)$\}$
\end{itemize}

\textbf{6.2 In terms of decision problems:}
\begin{itemize}
    \setlength\itemsep{-0.4em}
    \item DTIME($T(n)$)  = $\{ D |$ D is a decision problem $\wedge$ L(D) has time complexity T(n)$\}$
    \item DSPACE($S(n)$) = $\{ D |$ D is a decision problem $\wedge$ L(D) has space complexity S(n)$\}$
    \item NTIME($T(n)$)  = $\{ D |$ D is a decision problem $\wedge$ L(D) has nondet. time complexity T(n)$\}$
    \item DSPACE($S(n)$) = $\{ D |$ D is a decision problem $\wedge$ L(D) has nondet. space complexity S(n)$\}$
\end{itemize}

\textbf{6.3 Relations between different complexity classes:}
\begin{itemize}
    \setlength\itemsep{-0.4em}
    \item DTIME($T(n)$) $\subseteq$ DSPACE($T(n)$) \small{i.e. What can be solved in time $O(T(n))$, can also be solved on space $O(T(n))$}
    \item $L \in$ DSPACE($S(n)$) $\wedge S(n) \geq log_2 n \Rightarrow \exists c: L \in DTIME(c^{S(n)})$ \small{i.e. What can be solved nondeterminstically in space $O(S(n))$, can be solved deterministically in (at most) time $O(c^{S(n)})$}
    \item $L \in NTIME (T(n)) \Rightarrow \exists c: L \in DTIME(c^{T(n)})$ \small{i.e What can be solved nondeterminstically in time $O(T(n))$, can be solved deterministically in (at most) time $O(c^{T(n)})$}\\
        Consequentely, the substitution of nondeterministic algorithm with a deterministic one causes at most \textit{exponential} increase in the \textbf{time} required to (deterministically) solve a problem.
    \item $NSPACE(S(n)) \subseteq DSPACE(S^2(n))$, if $S(n) \geq log_2 n \wedge S(n)$ is \textit{"well behaved"} \small{i.e What can be solved nondeterminstically on space $O(S(n))$, can also be solved deterministically on space $O(S^2(n))$} \\
    Consequentely, the substitution of nondeterministic algorithm with a deterministic one causes at most \textit{quadratic} increase in the \textbf{space} required to (deterministically) solve a problem.
\end{itemize}

\textbf{6.4 Define P, NP, PSPACE, NPSPACE:}
\begin{itemize}
    \setlength\itemsep{-0.4em}
    \item $P = \cup_{i \geq 1}$ DTIME($n^i$) is the class of all decision problems \textbf{deterministically} solvable in \textit{polynomial time}.
    \item $NP = \cup_{i \geq 1}$ NTIME($n^i$) is the class of all decision problems \textbf{nondeterministically} solvable in \textit{polynomial time}.
    \item $PSPACE = \cup_{i \geq 1}$ DSPACE($n^i$) is the class of all decision problems \textbf{deterministically} solvable on \textit{polynomial space}.
    \item $NPSPACE= \cup_{i \geq 1}$ NSPACE($n^i$) is the class of all decision problems \textbf{nondeterministically} solvable on \textit{polynomial space}.
\end{itemize}

\textbf{6.5 Relations between P, NP, PSPACE, NPSPACE:}
\begin{center}
    $P \subseteq NP \subseteq PSPACE = NPSPACE$
\end{center}
Proof:
\begin{itemize}
    \setlength\itemsep{-0.4em}
    \item $P \subseteq NP \rightarrow$ Every deterministic TM of polynomial time complexity can be viewed as a (trivial) nondeterministic TM of the same complexity.
    \item NP $\subseteq$ PSPACE $\rightarrow$ If $L \in NP$, then $\exists k$ such that $L \in NTIME(n^k)$. So $L \in$ NSPACE$(n^k)$, and hence $L \in$ DSPACE$(n^{2k})$. Therefore $L \in $ PSPACE.
    \item (PSPACE = NPSPACE) $\rightarrow$ Trivially, PSPACE $\subseteq$ NPSPACE. \small{The opposite direction:
    NPSPACE = (def) = $\cup$ NSPACE($n^i$) $\subseteq$ (by Savitch) $\subseteq \cup$ DSPACE($n^j$) $\subseteq$ PSPACE}
 \end{itemize}

 \textbf{6.6 NP-complete $\&$ NP-hard problems}\\
 \textbf{NP-hard:} $D \leq^p D^*$, for every $D \in NP$.\\
 \textbf{NP-complete:} $D^* \in NP \wedge D \leq^p D^*$, for every $D \in NP$.\\
 Hence, $D^*$ is NP-complete if $D^*$ is in NP and $D^*$ is NP-hard.

\end{multicols}
\end{document} 