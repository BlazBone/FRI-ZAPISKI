\documentclass{article}
\usepackage[margin=0.2cm]{geometry}
\usepackage{amsmath}
\usepackage{multicol}
\usepackage{lipsum}% dummy text

\setlength{\columnseprule}{0.4pt}

\begin{document}
\begin{multicols}{3}

\section{\underline{Vektorji in matrike}}

\textbf{1.1} Vektor je \textit{urejena n-terica stevil}, ki jo obicajno
zapisemo kot stolpec\smallskip
\begin{center}
    $\vec{x}$ =
    $\begin{bmatrix}
        x_{1}\\
        \vdots \\
        x_{n}\\
    \end{bmatrix}$
\end{center}

\textbf{1.2} Produkt \textit{vektorja} $\vec{x}$ s skalarjem $\alpha$ je vektor
\begin{center}
    $\alpha \vec{x}$ =
    $\alpha$
    $\begin{bmatrix}
        x_{1}\\
        \vdots \\
        x_{n}\\
    \end{bmatrix}$ =
    $\begin{bmatrix}
        \alpha x_{1}\\
        \vdots \\
        \alpha x_{n}\\
    \end{bmatrix}$
\end{center}

\textbf{1.3} Vsota \textit{vektorjev} $\vec{x}$ in $\vec{y}$ je vektor
\begin{center}
    $\vec{x} + \vec{y} = 
    \begin{bmatrix}
        x_{1}\\
        \vdots \\
        x_{n}\\
    \end{bmatrix} +
    \begin{bmatrix}
        y_{1}\\
        \vdots \\
        y_{n}\\
    \end{bmatrix} =
    \begin{bmatrix}
        x_{1}  +  y_{1}\\
        \vdots\\
        x_{n} + y_{n}\\
    \end{bmatrix} 
    $
\end{center}

\textbf{1.4} Nicelni vektor $\vec{0}$ je tisti vektor, za katerega
je $\vec{a} + \vec{0} = \vec{0} + \vec{a} = \vec{a}$ za vsak vektor
$\vec{a}$. Vse komponente nicelnega vektorja so enake 0. Vsakemu vektorju
$\vec{a}$ priprada nasprotni vektor -$\vec{a}$, tako da je $\vec{a} + (-\vec{a}) = \vec{0}$
Razlika vektorjev $\vec{a}$ in $\vec{b}$ je vsota $\vec{a} + (-\vec{b})$ in jo
navadno zapisemo kot  $\vec{a} - \vec{b}$.

\textbf{Lastnosti vektorske vsote}
\begin{itemize}
    \item $\vec{a} + \vec{b} = \vec{b} + \vec{a}$ (komutativnost)
    \item $\vec{a} + (\vec{b} + \vec{c}) = (\vec{a} + \vec{b}) + \vec{c}$ (asociativnost)
    \item $a(\vec{a} + \vec{b}) = a\vec{a} + a\vec{b}$ (distributivnost)
\end{itemize}

\textbf{1.5} Linearna kombinacija vektorjev $\vec{x}$ in $\vec{y}$ je vsota
\begin{center}
    $a\vec{x} + b\vec{y}$
\end{center}

\textbf{1.6} Skalarni produkt vektorjev\\
\begin{center}
    $\begin{bmatrix} 
        x_{1}\\ 
        \vdots\\ 
        x_{n}\\
    \end{bmatrix}$ in
    $\begin{bmatrix} 
        y_{1}\\ 
        \vdots\\ 
        y_{n}\\
    \end{bmatrix}$ je stevilo    
\end{center}
\begin{center}
    $\vec{x} \cdot \vec{y} = x_{1}y_{1} + x_{2}y_{2} + \dots + x_{n}y_{n}$
\end{center}

\textbf{Lastnosti skalarnega produkta}
\begin{itemize}
    \item $\vec{x} \cdot \vec{y} = \vec{y} \cdot \vec{x}$ (komutativnost)
    \item $\vec{x} \cdot (\vec{y} + \vec{z}) = \vec{x} \cdot \vec{y} + \vec{x} \cdot \vec{z}$ (aditivnost)
    \item $\vec{x} \cdot (a \vec{y}) = a(\vec{x} \cdot \vec{y}) = (a \vec{x}) \cdot \vec{y}$ (homogenost)
    \item $\forall \vec{x}$ \textit{velja} $\vec{x} \cdot \vec{x} \geq 0$
\end{itemize}

\textbf{1.7} Dolzina vektorja $\vec{x}$ je
\begin{center}
    $||\vec{x}|| = \sqrt{\vec{x} \cdot \vec{x}}$
\end{center}

\textbf{1.8} Enotski vektor je vektor z dolzino 1.

\textbf{1.9} Za poljubna vektorja $\vec{u}, \vec{v} \in R^{n}$ velja:
\begin{center}
    $|\vec{u} \cdot \vec{v}| \leq ||\vec{u}||||\vec{v}||$.
\end{center}

\textbf{1.10} Za poljubna vektorja $\vec{u}, \vec{v} \in R^{n}$ velja:
\begin{center}
    $||\vec{u} + \vec{v}|| \leq ||\vec{u}||+||\vec{v}||$.
\end{center}

\textbf{1.11} Vektorja $\vec{x}$ in $\vec{y}$ sta ortogonalna
(ali pravokotna) natakno takrat, kadar je
\begin{center}
    $\vec{x} \cdot \vec{y} = $ 0    
\end{center}

\textbf{1.12} Ce je $\phi$ kot med vektorjema $\vec{x}$ in $\vec{y}$, potem je
\begin{center}
    $\dfrac{\vec{x} \cdot \vec{y}}{||\vec{x}|| ||\vec{y}||} =
    \cos \phi$
\end{center}

\textbf{1.13} Vektorski produkt:
\begin{center}
    $\vec{a} \times \vec{b} = (a_{2}b_{3} - a_{3}b_{2}) \textbf{i}$ +
    $(a_{3}b_{1} - a_{1}b_{3}) \textbf{j} + (a_{1}b_{2} - a_{2}b_{1}) \textbf{k}$
\end{center}

\textbf{Lastnosti vektorskega produkta}
\begin{itemize}
    \item $\vec{a} \times (\vec{b} + \vec{c}) = \vec{a} \times \vec{b} + \vec{a} \times \vec{c}$ (aditivnost)
    \item $\vec{b} \times \vec{a} = -\vec{a} \times \vec{b}$ (!komutativnost)
    \item $ (a \vec{a}) \times \vec{b} = a(\vec{a} \times \vec{b}) =  \vec{a} \times (a \vec{b})$ (homogenost)
    \item $\vec{a} \times \vec{a} = 0$
    \item $\vec{a} \times \vec{b}$  \textit{je}  $\perp$ \textit{na vektorja} $\vec{a}$ \textit{in} $\vec{b}$
    \item $||\vec{a} \times \vec{b}|| = ||\vec{a}|| ||\vec{b}|| \sin \phi$
    \item Dolzina vektorskega produkta je ploscina paralelograma, katerega vektorja oklepata 
\end{itemize}

\textbf{1.14} Mesani produkt($\vec{a}, \vec{b}, \vec{c}$) vektorjev
$\vec{a}, \vec{b}$ in $\vec{c}$ v $R^{3}$ je skalarni produkt vektorjev
$\vec{a} \times \vec{b}$ in $\vec{c}$:
\begin{center}
    $(\vec{a}, \vec{b}, \vec{c}) = (\vec{a} \times \vec{b})\cdot \vec{c}$
\end{center}

\textbf{Lastnosti mesanega produkta}
\begin{itemize}
    \item $(\vec{a}, \vec{b}, \vec{c}) = (\vec{b}, \vec{c}, \vec{a}) = (\vec{c}, \vec{a}, \vec{b})$
    \item $(x\vec{a}, \vec{b}, \vec{c}) = x(\vec{a}, \vec{b}, \vec{c})$ (homogenost)
    \item $(\vec{a}, \vec{u} + \vec{v}, \vec{c}) = (\vec{a}, \vec{u}, \vec{c}) + (\vec{a}, \vec{v}, \vec{c})$
    \item Absolutna vrednost mesanega produkta ($\vec{a}, \vec{b}, \vec{c}$) je enaka prostornini paralepipeda
\end{itemize}

\textbf{Razdalje}\\
Razdalja od tocke $P$ do ravnine, v kateri lezi tocka $A$ :
\begin{center}
    $\cos\phi = \dfrac{\vec{n} \cdot ( \vec{r_{P}} - \vec{r_{A}})} {||\vec{n}|| ||\vec{r_{P}} - \vec{r_{A}}||}$ oz.
    $d = |\dfrac{\vec{n}}{||\vec{n}||} ( \vec{r_{P}} - \vec{r_{A}})|$
\end{center}
Razdalja od tocke $P$ do premice, katera gre skozi tocko $A$:
\begin{center}
    $d = \dfrac{||\vec{e} \times ( \vec{r_{P}} - \vec{r_{A}})||}{||\vec{e}||}$
\end{center}

\textbf{1.15} Matrika dimenzije $m \times n$ je tabela $m \times n$ stevil, urejenih
v $m$ vrstic in $n$ stolpcev:
\begin{center}
    $A^{m \times n} =$
    $\begin{bmatrix}
        x_{11} & x_{12} & x_{13} & \dots  & x_{1n} \\
        x_{21} & x_{22} & x_{23} & \dots  & x_{2n} \\
        \vdots & \vdots & \vdots & \ddots & \vdots \\
        x_{m1} & x_{m2} & x_{m3} & \dots  & x_{mn}
    \end{bmatrix}$
\end{center}

\textbf{1.16} Matrika, katere elementi so enaki nic povsod
zunaj glavne diagonale, se imenuje diagonalna matrika. Za
diagonalno matriko je $a_{ij} = 0$, kadarkoli velja $i \neq j$

\textbf{1.17} Matrika $A^{n \times n}$ je spodnjetrikotna, kadar
so vsi elementi nad glavno diagonalo enaki 0:
\begin{center}
    $a_{ij} = 0$  \textit{kadar je} $i < j$
\end{center}

\textbf{1.18} Matrika $A^{n \times n}$ je zgornjetrikotna, kadar
so vsi elementi pod glavno diagonalo enaki 0:
\begin{center}
    $a_{ij} = 0$  \textit{kadar je} $i > j$
\end{center}

\textbf{1.19} Matrika je trikotna, ce je zgornjetrikotna ali spodnjetrikotna.

\textbf{1.20} Dve matriki $A$ in $B$ sta enaki natanko takrat,
kadar imata enaki dimenziji in kadar so na istih mestih v obeh
matrikah enaki elementi:
\begin{center}
    $A^{m \times n} = B^{p \times q} \implies m=p$ in $n=q$,\\
    $a_{ij} = b_{ij}$ \textit{za vsak} $i= 1,...,m$ in $j=1,...,n$ 
\end{center}

\textbf{1.21} Produkt matrike s skalarjem dobimo tako, da 
vsak element matrike pomnozimo s $skalarjem$
\begin{center}
    $aA^{m \times n} =$
    $\begin{bmatrix}
        ax_{11} & ax_{12} & ax_{13} & \dots  & ax_{1n} \\
        ax_{21} & ax_{22} & ax_{23} & \dots  & ax_{2n} \\
        \vdots  & \vdots  & \vdots  & \ddots  & \vdots \\
        ax_{m1} & ax_{m2} & ax_{m3} & \dots  & ax_{mn}
    \end{bmatrix}$
\end{center}

\textbf{1.22} Vsoto dveh matrik enake dimenzije dobimo tako,
da sestejemo istolezne elemente obeh matrik:
\begin{center}
    $A + B =$
    $\begin{bmatrix}
        a_{11} + b_{11} & ax_{12} + b_{12}  & \dots  & ax_{1n} + b_{1n} \\
        a_{21} + b_{21} & ax_{22} + b_{22}  & \dots  & ax_{2n} + b_{2n}\\
        \vdots          & \vdots            & \ddots & \vdots \\
        a_{m1} + b_{m1} & ax_{m2} + b_{m3} & \dots  & ax_{mn} + b_{mn}
    \end{bmatrix}$
\end{center}

\textbf{Osnovne matricne operacije}
\begin{itemize}
    \item $A + B = B + A$ (komutativnost)
    \item $(A + B) + C = A + (B + C)$ (asociativnost)
    \item $a(A + B) = aA + aB$ (mnozenje s skalarjem)
    \item $A + (-A) = 0$
    \item $x(yA) = (xy)A$ \textit{in} $1 \cdot A = A$
\end{itemize}

\textbf{1.23} Transponirana matrika k matriki A reda $m \times n$
je matrika reda $n \times m$
\begin{center}
    $A =$
    $\begin{bmatrix}
        x_{11} & x_{12} & \dots  & x_{1n} \\
        x_{21} & x_{22} & \dots  & x_{2n} \\
        \vdots & \vdots & \ddots & \vdots \\
        x_{m1} & x_{m2} & \dots  & x_{mn}
    \end{bmatrix}$\\
    \smallskip
    $A^{T} =$
    $\begin{bmatrix}
        x_{11} & x_{21} & \dots  & x_{m1} \\
        x_{12} & x_{22} & \dots  & x_{m2} \\
        \vdots & \vdots & \ddots & \vdots \\
        x_{1n} & x_{2n} & \dots  & x_{mn}
    \end{bmatrix}$
\end{center}

\textbf{Lastnosti transponiranja matrik}
\begin{itemize}
    \item $(A + B)^{T} = A^{T} + B^{T}$
    \item $(xA)^{T} = xA^{T}$
    \item $(A^{T})^{T} = A$
\end{itemize}

\textbf{1.24} Produkt matrike A in vektorja $\vec{x}$ je
linearna kombinacija stolpcev matrike A, utezi linearne
kombinacije so komponente vektorja $\vec{x}$:
\begin{center}
    $A \vec{x} =
    \begin{bmatrix}
                &         & \\
        \vec{u} & \vec{v} & \vec{w} \\
                &         & \\
    \end{bmatrix}
    \cdot
    \begin{bmatrix}
        a\\
        b\\
        c
    \end{bmatrix} =$
    $a\vec{u} + b\vec{v} + c\vec{w}$
\end{center}

\textbf{1.25} Produkt vrstice $\vec{x}$ z matriko A je
linearna kombinacija vrstic matrike A, koeficienti linearne
kombinacije so komponente vrstice $\vec{y}$:
\begin{center}
    $\vec{y} \cdot A =
    \begin{bmatrix}
        y_{1}, y_{2}, y_{3}
    \end{bmatrix} \cdot
    \begin{bmatrix}
        \vec{u}\\
        \vec{v}\\
        \vec{w}
    \end{bmatrix} =
    \begin{bmatrix}
        y_{1}\vec{u}\\
        y_{2}\vec{v}\\
        y_{3}\vec{w}
    \end{bmatrix}
    $
\end{center}

\textbf{1.26} Produkt matrik A in B je matrika, katere stolpci
so zaporedoma produkti matrike A s stolpci matrike B:
\begin{center}
    $AB = A
    \begin{bmatrix}
        b_{1}, b_{2}, \dots ,b_{n}
    \end{bmatrix} =
    \begin{bmatrix}
        Ab_{1}, Ab_{2}, \dots ,Ab_{n}
    \end{bmatrix}
    $
\end{center}

\textbf{1.27} Element $c_{ij}$ v $i-ti$ vrstici in $j-tem$ stolpcu
produkta C = AB je skalarni produkt $i-te$ vrstice A in $j-tega$
stolpca matrike B
\begin{center}
    $c_{ij} =
    \sum_{k=1}^{n} a_{ik}b_{kj}
    $
\end{center}

\textbf{1.28} Produkt matrik A in B je matrika, katere vrstice
so zaporedoma produkti vrstic matrike A z matriko B:
\begin{center}
    $
    \begin{bmatrix}
        i-ta\; vrstica\; A
    \end{bmatrix}B =
    \begin{bmatrix}
        i-ta\; vrstica\; AB
    \end{bmatrix}
    $
\end{center}

\textbf{Lastnosti matricnega produkta}
\begin{itemize}
    \item $AB \neq BA$ (!komutativnost)
    \item $(xA)B = x(AB) = A(xB)$ (homogenost)
    \item $C(A + B) = CA + CB$ (distributivnost)
    \item $A(BC) = (AB)C$ (asociativnost)
    \item $(AB)^{T} = B^{T}A^{T}$
\end{itemize}

\textbf{1.29} Vrstice matrike A z $n$ stolpci naj bodo
$a^{1}, \dots, a^{n}$, stolpci matrike B z $n$ vrsticami pa
$a_{1}, \dots, b_{n}$. Potem je
\begin{center}
    $AB = a^{1}b_{1} + \dots + a^{n}b_{n}$
\end{center}

\textbf{1.30} Ce delitev na bloke v matriki A ustreza delitvi v matirki B,
potem lahko matriki pomnozimo blocno:
\begin{center}
    $\begin{bmatrix}
        A_{11} & A_{12}\\
        A_{21} & A_{22}
    \end{bmatrix}
    \begin{bmatrix}
        B_{11} & B_{12}\\
        B_{21} & B_{22}
    \end{bmatrix} =
    \begin{bmatrix}
        A_{11}B_{11} + A_{12}B_{21} & A_{11}B_{12} + A_{12}B_{22}\\
        A_{21}B_{11} + A_{22}B_{21} & A_{21}B_{12} + A_{22}B_{22}
    \end{bmatrix}$
\end{center}

\section{Sistemi linearnih enacb}

\end{multicols}
\end{document}
