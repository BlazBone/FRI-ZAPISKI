\documentclass{article}
\usepackage[margin=0.15cm]{geometry}
\usepackage{amsmath}
\usepackage{multicol}
\usepackage{hyperref}

\setlength{\columnseprule}{0.5pt}

\begin{document}

\begin{center}
    {\huge \href{https://github.com/spagnoloG/LA-FRI}{LA/FRI} - Cheat Sheet \par}
\end{center}

\begin{multicols}{3}

\section{\underline{Vektorji in matrike}}

\textbf{1.1} Vektor je \textit{urejena n-terica stevil}, ki jo obicajno
zapisemo kot stolpec\smallskip
\begin{center}
    $\vec{x}$ =
    $\begin{bmatrix}
        x_{1}\\
        \vdots \\
        x_{n}\\
    \end{bmatrix}$
\end{center}

\textbf{1.2} Produkt \textit{vektorja} $\vec{x}$ s skalarjem $\alpha$ je vektor
\begin{center}
    $\alpha \vec{x}$ =
    $\alpha$
    $\begin{bmatrix}
        x_{1}\\
        \vdots \\
        x_{n}\\
    \end{bmatrix}$ =
    $\begin{bmatrix}
        \alpha x_{1}\\
        \vdots \\
        \alpha x_{n}\\
    \end{bmatrix}$
\end{center}

\textbf{1.3} Vsota \textit{vektorjev} $\vec{x}$ in $\vec{y}$ je vektor
\begin{center}
    $\vec{x} + \vec{y} = 
    \begin{bmatrix}
        x_{1}\\
        \vdots \\
        x_{n}\\
    \end{bmatrix} +
    \begin{bmatrix}
        y_{1}\\
        \vdots \\
        y_{n}\\
    \end{bmatrix} =
    \begin{bmatrix}
        x_{1}  +  y_{1}\\
        \vdots\\
        x_{n} + y_{n}\\
    \end{bmatrix} 
    $
\end{center}

\textbf{1.4} Nicelni vektor $\vec{0}$ je tisti vektor, za katerega
je $\vec{a} + \vec{0} = \vec{0} + \vec{a} = \vec{a}$ za vsak vektor
$\vec{a}$. Vse komponente nicelnega vektorja so enake 0. Vsakemu vektorju
$\vec{a}$ priprada nasprotni vektor -$\vec{a}$, tako da je $\vec{a} + (-\vec{a}) = \vec{0}$
Razlika vektorjev $\vec{a}$ in $\vec{b}$ je vsota $\vec{a} + (-\vec{b})$ in jo
navadno zapisemo kot  $\vec{a} - \vec{b}$.

\textbf{Lastnosti vektorske vsote}
\begin{itemize}
    \item $\vec{a} + \vec{b} = \vec{b} + \vec{a}$ (komutativnost)
    \item $\vec{a} + (\vec{b} + \vec{c}) = (\vec{a} + \vec{b}) + \vec{c}$ (asociativnost)
    \item $a(\vec{a} + \vec{b}) = a\vec{a} + a\vec{b}$ (distributivnost)
\end{itemize}

\textbf{1.5} Linearna kombinacija vektorjev $\vec{x}$ in $\vec{y}$ je vsota
\begin{center}
    $a\vec{x} + b\vec{y}$
\end{center}

\textbf{1.6} Skalarni produkt vektorjev\\
\begin{center}
    $\begin{bmatrix} 
        x_{1}\\ 
        \vdots\\ 
        x_{n}\\
    \end{bmatrix}$ in
    $\begin{bmatrix} 
        y_{1}\\ 
        \vdots\\ 
        y_{n}\\
    \end{bmatrix}$ je stevilo    
\end{center}
\begin{center}
    $\vec{x} \cdot \vec{y} = x_{1}y_{1} + x_{2}y_{2} + \dots + x_{n}y_{n}$
\end{center} \textit{alternativno:}
\begin{center}
    $\vec{x} \cdot \vec{y} = ||\vec{x}|| ||\vec{y}|| \cos \phi$
\end{center}

\textbf{Lastnosti skalarnega produkta}
\begin{itemize}
    \item $\vec{x} \cdot \vec{y} = \vec{y} \cdot \vec{x}$ (komutativnost)
    \item $\vec{x} \cdot (\vec{y} + \vec{z}) = \vec{x} \cdot \vec{y} + \vec{x} \cdot \vec{z}$ (aditivnost)
    \item $\vec{x} \cdot (a \vec{y}) = a(\vec{x} \cdot \vec{y}) = (a \vec{x}) \cdot \vec{y}$ (homogenost)
    \item $\forall \vec{x}$ \textit{velja} $\vec{x} \cdot \vec{x} \geq 0$
\end{itemize}

\textbf{1.7} Dolzina vektorja $\vec{x}$ je
\begin{center}
    $||\vec{x}|| = \sqrt{\vec{x} \cdot \vec{x}}$
\end{center}

\textbf{1.8} Enotski vektor je vektor z dolzino 1.

\textbf{1.9} Za poljubna vektorja $\vec{u}, \vec{v} \in R^{n}$ velja:
\begin{center}
    $|\vec{u} \cdot \vec{v}| \leq ||\vec{u}||||\vec{v}||$.
\end{center}

\textbf{1.10} Za poljubna vektorja $\vec{u}, \vec{v} \in R^{n}$ velja:
\begin{center}
    $||\vec{u} + \vec{v}|| \leq ||\vec{u}||+||\vec{v}||$.
\end{center}

\textbf{1.11} Vektorja $\vec{x}$ in $\vec{y}$ sta ortogonalna
(ali pravokotna) natakno takrat, kadar je
\begin{center}
    $\vec{x} \cdot \vec{y} = $ 0    
\end{center}

\textbf{1.12} Ce je $\phi$ kot med vektorjema $\vec{x}$ in $\vec{y}$, potem je
\begin{center}
    $\dfrac{\vec{x} \cdot \vec{y}}{||\vec{x}|| ||\vec{y}||} =
    \cos \phi$
\end{center}

\textbf{1.13} Vektorski produkt:
\begin{center}
    $\vec{a} \times \vec{b} = (a_{2}b_{3} - a_{3}b_{2}) \textbf{i}$ +
    $(a_{3}b_{1} - a_{1}b_{3}) \textbf{j} + (a_{1}b_{2} - a_{2}b_{1}) \textbf{k}$
\end{center}

\textbf{Lastnosti vektorskega produkta}
\begin{itemize}
    \item $\vec{a} \times (\vec{b} + \vec{c}) = \vec{a} \times \vec{b} + \vec{a} \times \vec{c}$ (aditivnost)
    \item $\vec{b} \times \vec{a} = -\vec{a} \times \vec{b}$ (!komutativnost)
    \item $ (a \vec{a}) \times \vec{b} = a(\vec{a} \times \vec{b}) =  \vec{a} \times (a \vec{b})$ (homogenost)
    \item $\vec{a} \times \vec{a} = 0$
    \item $\vec{a} \times \vec{b}$  \textit{je}  $\perp$ \textit{na vektorja} $\vec{a}$ \textit{in} $\vec{b}$
    \item $||\vec{a} \times \vec{b}|| = ||\vec{a}|| ||\vec{b}|| \sin \phi$
    \item Dolzina vektorskega produkta je ploscina paralelograma, katerega vektorja oklepata 
\end{itemize}

\textbf{1.14} Mesani produkt($\vec{a}, \vec{b}, \vec{c}$) vektorjev
$\vec{a}, \vec{b}$ in $\vec{c}$ v $R^{3}$ je skalarni produkt vektorjev
$\vec{a} \times \vec{b}$ in $\vec{c}$:
\begin{center}
    $(\vec{a}, \vec{b}, \vec{c}) = (\vec{a} \times \vec{b})\cdot \vec{c}$
\end{center}

\textbf{Lastnosti mesanega produkta}
\begin{itemize}
    \item $(\vec{a}, \vec{b}, \vec{c}) = (\vec{b}, \vec{c}, \vec{a}) = (\vec{c}, \vec{a}, \vec{b})$
    \item $(x\vec{a}, \vec{b}, \vec{c}) = x(\vec{a}, \vec{b}, \vec{c})$ (homogenost)
    \item $(\vec{a}, \vec{u} + \vec{v}, \vec{c}) = (\vec{a}, \vec{u}, \vec{c}) + (\vec{a}, \vec{v}, \vec{c})$
    \item Absolutna vrednost mesanega produkta ($\vec{a}, \vec{b}, \vec{c}$) je enaka prostornini paralepipeda
\end{itemize}

\textbf{Razdalje}\\
Razdalja od tocke $P$ do ravnine, v kateri lezi tocka $A$ :
\begin{center}
    $\cos\phi = \dfrac{\vec{n} \cdot ( \vec{r_{P}} - \vec{r_{A}})} {||\vec{n}|| ||\vec{r_{P}} - \vec{r_{A}}||}$ oz.
    $d = |\dfrac{\vec{n}}{||\vec{n}||} ( \vec{r_{P}} - \vec{r_{A}})|$
\end{center}
Razdalja od tocke $P$ do premice, katera gre skozi tocko $A$:
\begin{center}
    $d = \dfrac{||\vec{e} \times ( \vec{r_{P}} - \vec{r_{A}})||}{||\vec{e}||}$
\end{center}

\textbf{Projekcije vekotrjev}\\
Naj bo $proj_{\vec{a}}\vec{b} = \vec{x}$ projekcija vektorja $\vec{b}$ na vektor $\vec{a}$.
Izracunamo jo po sledeci formuli:
\begin{center}
    \begin{math}
        proj_{\vec{a}}\vec{b} = \frac{\vec{a}\vec{b}}{\vec{a}\vec{a}} \vec{a}
    \end{math}
\end{center}

\textbf{1.15} Matrika dimenzije $m \times n$ je tabela $m \times n$ stevil, urejenih
v $m$ vrstic in $n$ stolpcev:
\begin{center}
    $A^{m \times n} =$
    $\begin{bmatrix}
        x_{11} & x_{12} & x_{13} & \dots  & x_{1n} \\
        x_{21} & x_{22} & x_{23} & \dots  & x_{2n} \\
        \vdots & \vdots & \vdots & \ddots & \vdots \\
        x_{m1} & x_{m2} & x_{m3} & \dots  & x_{mn}
    \end{bmatrix}$
\end{center}

\textbf{1.16} Matrika, katere elementi so enaki nic povsod
zunaj glavne diagonale, se imenuje diagonalna matrika. Za
diagonalno matriko je $a_{ij} = 0$, kadarkoli velja $i \neq j$

\textbf{1.17} Matrika $A^{n \times n}$ je spodnjetrikotna, kadar
so vsi elementi nad glavno diagonalo enaki 0:
\begin{center}
    $a_{ij} = 0$  \textit{kadar je} $i < j$
\end{center}

\textbf{1.18} Matrika $A^{n \times n}$ je zgornjetrikotna, kadar
so vsi elementi pod glavno diagonalo enaki 0:
\begin{center}
    $a_{ij} = 0$  \textit{kadar je} $i > j$
\end{center}

\textbf{1.19} Matrika je trikotna, ce je zgornjetrikotna ali spodnjetrikotna.

\textbf{1.20} Dve matriki $A$ in $B$ sta enaki natanko takrat,
kadar imata enaki dimenziji in kadar so na istih mestih v obeh
matrikah enaki elementi:
\begin{center}
    $A^{m \times n} = B^{p \times q} \implies m=p$ in $n=q$,\\
    $a_{ij} = b_{ij}$ \textit{za vsak} $i= 1,...,m$ in $j=1,...,n$ 
\end{center}

\textbf{1.21} Produkt matrike s skalarjem dobimo tako, da 
vsak element matrike pomnozimo s $skalarjem$
\begin{center}
    $aA^{m \times n} =$
    $\begin{bmatrix}
        ax_{11} & ax_{12} & ax_{13} & \dots  & ax_{1n} \\
        ax_{21} & ax_{22} & ax_{23} & \dots  & ax_{2n} \\
        \vdots  & \vdots  & \vdots  & \ddots  & \vdots \\
        ax_{m1} & ax_{m2} & ax_{m3} & \dots  & ax_{mn}
    \end{bmatrix}$
\end{center}

\textbf{1.22} Vsoto dveh matrik enake dimenzije dobimo tako,
da sestejemo istolezne elemente obeh matrik:
\begin{center}
    $A + B =$
    $\begin{bmatrix}
        a_{11} + b_{11} & ax_{12} + b_{12}  & \dots  & ax_{1n} + b_{1n} \\
        a_{21} + b_{21} & ax_{22} + b_{22}  & \dots  & ax_{2n} + b_{2n}\\
        \vdots          & \vdots            & \ddots & \vdots \\
        a_{m1} + b_{m1} & ax_{m2} + b_{m3} & \dots  & ax_{mn} + b_{mn}
    \end{bmatrix}$
\end{center}

\textbf{Osnovne matricne operacije}
\begin{itemize}
    \item $A + B = B + A$ (komutativnost)
    \item $(A + B) + C = A + (B + C)$ (asociativnost)
    \item $a(A + B) = aA + aB$ (mnozenje s skalarjem)
    \item $A + (-A) = 0$
    \item $x(yA) = (xy)A$ \textit{in} $1 \cdot A = A$
\end{itemize}

\textbf{1.23} Transponirana matrika k matriki A reda $m \times n$
je matrika reda $n \times m$
\begin{center}
    $A =$
    $\begin{bmatrix}
        x_{11} & x_{12} & \dots  & x_{1n} \\
        x_{21} & x_{22} & \dots  & x_{2n} \\
        \vdots & \vdots & \ddots & \vdots \\
        x_{m1} & x_{m2} & \dots  & x_{mn}
    \end{bmatrix}$\\
    \smallskip
    $A^{T} =$
    $\begin{bmatrix}
        x_{11} & x_{21} & \dots  & x_{m1} \\
        x_{12} & x_{22} & \dots  & x_{m2} \\
        \vdots & \vdots & \ddots & \vdots \\
        x_{1n} & x_{2n} & \dots  & x_{mn}
    \end{bmatrix}$
\end{center}

\textbf{Lastnosti transponiranja matrik}
\begin{itemize}
    \item $(A + B)^{T} = A^{T} + B^{T}$
    \item $(xA)^{T} = xA^{T}$
    \item $(A^{T})^{T} = A$
\end{itemize}

\textbf{1.24} Produkt matrike A in vektorja $\vec{x}$ je
linearna kombinacija stolpcev matrike A, utezi linearne
kombinacije so komponente vektorja $\vec{x}$:
\begin{center}
    $A \vec{x} =
    \begin{bmatrix}
                &         & \\
        \vec{u} & \vec{v} & \vec{w} \\
                &         & \\
    \end{bmatrix}
    \cdot
    \begin{bmatrix}
        a\\
        b\\
        c
    \end{bmatrix} =$
    $a\vec{u} + b\vec{v} + c\vec{w}$
\end{center}

\textbf{1.25} Produkt vrstice $\vec{x}$ z matriko A je
linearna kombinacija vrstic matrike A, koeficienti linearne
kombinacije so komponente vrstice $\vec{y}$:
\begin{center}
    $\vec{y} \cdot A =
    \begin{bmatrix}
        y_{1}, y_{2}, y_{3}
    \end{bmatrix} \cdot
    \begin{bmatrix}
        \vec{u}\\
        \vec{v}\\
        \vec{w}
    \end{bmatrix} =
    \begin{bmatrix}
        y_{1}\vec{u}\\
        y_{2}\vec{v}\\
        y_{3}\vec{w}
    \end{bmatrix}
    $
\end{center}

\textbf{1.26} Produkt matrik A in B je matrika, katere stolpci
so zaporedoma produkti matrike A s stolpci matrike B:
\begin{center}
    $AB = A
    \begin{bmatrix}
        b_{1}, b_{2}, \dots ,b_{n}
    \end{bmatrix} =
    \begin{bmatrix}
        Ab_{1}, Ab_{2}, \dots ,Ab_{n}
    \end{bmatrix}
    $
\end{center}

\textbf{1.27} Element $c_{ij}$ v $i-ti$ vrstici in $j-tem$ stolpcu
produkta C = AB je skalarni produkt $i-te$ vrstice A in $j-tega$
stolpca matrike B
\begin{center}
    $c_{ij} =
    \sum_{k=1}^{n} a_{ik}b_{kj}
    $
\end{center}

\textbf{1.28} Produkt matrik A in B je matrika, katere vrstice
so zaporedoma produkti vrstic matrike A z matriko B:
\begin{center}
    $
    \begin{bmatrix}
        i-ta\; vrstica\; A
    \end{bmatrix}B =
    \begin{bmatrix}
        i-ta\; vrstica\; AB
    \end{bmatrix}
    $
\end{center}

\textbf{Lastnosti matricnega produkta}
\begin{itemize}
    \item $AB \neq BA$ (!komutativnost)
    \item $(xA)B = x(AB) = A(xB)$ (homogenost)
    \item $C(A + B) = CA + CB$ (distributivnost)
    \item $A(BC) = (AB)C$ (asociativnost)
    \item $(AB)^{T} = B^{T}A^{T}$
\end{itemize}

\textbf{1.29} Vrstice matrike A z $n$ stolpci naj bodo
$a^{1}, \dots, a^{n}$, stolpci matrike B z $n$ vrsticami pa
$a_{1}, \dots, b_{n}$. Potem je
\begin{center}
    $AB = a^{1}b_{1} + \dots + a^{n}b_{n}$
\end{center}

\textbf{1.30} Ce delitev na bloke v matriki A ustreza delitvi v matirki B,
potem lahko matriki pomnozimo blocno:
\begin{center}
    $\begin{bmatrix}
        A_{11} & A_{12}\\
        A_{21} & A_{22}
    \end{bmatrix}
    \begin{bmatrix}
        B_{11} & B_{12}\\
        B_{21} & B_{22}
    \end{bmatrix} =
    \begin{bmatrix}
        A_{11}B_{11} + A_{12}B_{21} & A_{11}B_{12} + A_{12}B_{22}\\
        A_{21}B_{11} + A_{22}B_{21} & A_{21}B_{12} + A_{22}B_{22}
    \end{bmatrix}$
\end{center}

\textbf{1.31} Kvadratna matrika $I_{k}$ reda $k \times k$, ki ima vse diagonalne
elemente enake 1, vse ostale elemente pa 0 ima lastnost, da za vsako matriko A
reda $m \times n$ velja $AI_{n} = A$ in $I_{m}A = A$. Matrika $I_{k}$ se imenuje
enotska ali identicna matirka.
\begin{center}
    $I_{k}=
    \begin{bmatrix}
        1 & 0 & \hdots & 0\\
        0 & 1 & \hdots & 0 \\
        \vdots & \vdots & \ddots & \vdots\\
        0 & 0 & \hdots & 1
    \end{bmatrix} 
    $
\end{center}

\section{\underline{Sistemi linearnih enacb}}

\textbf{2.1} Kvadratna matrika A je obrnljiva, ce obstaja taka matrika
$A^{-1}$, da je
\begin{center}
    $AA^{-1} = I\;
    in\;
    A^{-1}A = I
    $
\end{center}
Matrika $A^{-1}$ (ce obstaja) se imenuje matriki A inverzna matrika.
Matrika, ki ni obrnljiva, je singularna. Matrika \textbf{NI} obrnljiva, kadar je
$rang(A) < n$ !

\textbf{2.2} Kvadratna matirka reda $n$ je obrnljiva natanko tedaj, ko pri
gaussovi eliminaciji dobimo $n$ pivotov.

\textbf{2.3} Vsaka obrnljiva matrika ima eno samo inverzno matriko.

\textbf{2.4} Inverzna matrika inverzne matrike $A^{-1}$ je matrika A
\begin{center}
    $(A^{-1})^{-1} = A$
\end{center}

\textbf{2.5} Ce je matrika A obrnljiva, potem ima sistem enacb
$A\vec{x} = \vec{b}$ edino resitev $\vec{x} = A^{-1} \vec{b}$

\textbf{2.6} Ce obstaja nenicelna resitev $\vec{x}$ enacbe $A\vec{x} = \vec{0}$,
matrika A ni obrnljiva(je singularna).

\textbf{2.7} Ce sta matirki A in B istega reda obrnljivi, je obrnljiv tudi
produkt $A \cdot B$ in
\begin{center}
    $(A \cdot B)^{-1} =
    B^{-1} \cdot A^{-1}
    $
\end{center}

\textbf{Pozor!} Pravilo
\begin{center}
    $(AB)^{p} = A^{p}B^{p}$
\end{center}
velja le v primeru, ko matriki A in B komutirata, torej $AB = BA$.

\textbf{2.8} Inverz transponirane matrike je transponirana matrika inverza
\begin{center}
    $(A^{T})^{-1} = (A^{-1})^{T}$
\end{center}

\textbf{2.9} Inverz diagonalne matrike z diagonalnimi elementi $a_{ii}$ je
diagonalna matrika, ki ima na diagonali elemente $a_{ii}^{-1}$
\begin{center}
    $\begin{bmatrix}
        a_{11} &        & 0\\
               & \ddots &\\
        0      &        & a_{nn}
    \end{bmatrix}=
    \begin{bmatrix}
        a_{11}^{-1} &        & 0\\
                    & \ddots &\\
        0           &        & a_{nn}^{-1}
    \end{bmatrix}
    $
\end{center}

\textbf{2.10} Za izracun inverza matrike A, uporabimo gausovo eliminacijo nad
matriko $\begin{bmatrix}A|I\end{bmatrix}$
\begin{center}
    $\begin{bmatrix}A|I\end{bmatrix} =
    \begin{bmatrix}I|A^{-1}\end{bmatrix}
    $
\end{center}

\textbf{2.11} Matrika A je simetricna $\Leftrightarrow A^{T} = A$. Za elemente
$a_{ij}$ simetricne matirke velja $a_{ij} = a_{ji}$.

\textbf{2.12} Ce je matrika A simetricna in obrnljiva, je tudi $A^{-1}$ simetricna.

\textbf{2.13} Ce je R poljubna (lahko tudi pravokotna) matrika, sta $R^{T}R$ in
$RR^{T}$ simetricni matriki.

\section{\underline{Vektorski prostori}}

\textbf{3.1} Realni vektorski prostor V je mnozica "vektorjev" skupaj z pravili za
\begin{itemize}
    \item sestevanje vektorjev,
    \item mnozenje vektorja z realnim stevilom (skalarjem)
\end{itemize}
Ce sta $\vec{x}$ in $\vec{y}$ poljubna vektorja v V, morajo biti v V tudi
\begin{itemize}
    \item vsota $\vec{x} + \vec{y}$ in 
    \item produkti $\alpha\vec{x}$ za vse $\alpha \in R$
\end{itemize}
V vektorskem prostoru V morajo biti tudi VSE linearne kombinacije
$\alpha\vec{x} + \beta\vec{y}$

\textbf{Pravila za operacije v vektorskih prostorih}\\
Operaciji sestevanja vektorjev in mnozenja vektorja s skalarjem v vektorskem prostoru
morajo zadoscati naslednjim pravilom:
\begin{itemize}
    \item $\vec{x} + \vec{y} = \vec{y} + \vec{x}$ (komutativnost)
    \item $\vec{x} + (\vec{y} + \vec{z}) = (\vec{x} + \vec{y}) + \vec{z}$ (asociativnost)
    \item obstaja en sam nenicelni vektor $\vec{0}$, da velja $\vec{x} + \vec{0} = \vec{x}$
    \item za vsak $\vec{x}$ obstaja natanko en $-\vec{x}$, da je $\vec{x} + (-\vec{x}) = \vec{0}$
    \item $1 \cdot \vec{x} = \vec{x}$
    \item $(\alpha\beta)\vec{x} = \alpha(\beta\vec{x})$
    \item $\alpha(\vec{x} + \vec{y}) = \alpha\vec{x} + \alpha\vec{y}$ (distributivnost)
    \item $(\alpha + \beta)\vec{x} = \alpha\vec{x} + \beta\vec{x}$
\end{itemize}

\textbf{3.2} Podmnozica U vektorskega prostora V je \textit{vektorski podprostor}, ce je za
vsak par vektorjev $\vec{x}$ in $\vec{y}$ iz U in vsako realno stevilo $\alpha$ tudi
\begin{itemize}
    \item $\vec{x} + \vec{y} \in U$ in
    \item $\alpha\vec{x} \in U$.
\end{itemize}

\textbf{3.3} Mnozica vektorjev U je vektorski podprostor natanko tedaj, ko je vsaka linearna
kombinacija vektorjev iz U tudi v U.

\textbf{Lastnosti vektorskih podprostorov}
\begin{itemize}
    \item Vsak vektorski podprostor nujno vsebuje nicelni vektor $\vec{0}$
    \item Presek dveh podprostorov vektorskega podprostora je tudi podprostor
\end{itemize}

\textbf{3.4} Stolpicni prostor C(A) matrike $A \in R^{m \times n}$ je tisti podprostor
vektorskega prostora $R^{m}$, ki vsebuje natanko vse linearne kombinacije stolpcev matrike A.\\
Izracunamo ga tako, da matriko A transponiramo in izvedemo operacijo gaussove eliminacije nad $A^{T}$. Vrstice katere ostanejo po gaussivi eliminaciji
so linearno neodvisni vektorji, kateri tvorijo stoplicni prostor matrike A, $C(A)$.
\textit{neformalno: linearna ogrinjaca stolpcev matrike (npr. ce imas 5 stolpcev pa lahko 2 zapises kot linearno kombinacijo ostalih 3 bo imel column space 3 elemente)}

\textbf{3.5} Sistem linearnih enacb $A\vec{x} = \vec{b}$ je reslijv natanko tedaj, ko je vektor
$\vec{b} \in C(A)$

\textbf{3.6} Naj bo matrika $A \in R^{m \times n}$. Mnozica resitev homogenega sistema linearnih
enacb je podprostor v vektorskem prostoru $R^{n}$.

\textbf{3.7} Mnozica vseh resitev sistema linearnih enacb $A\vec{x} = \vec{0}$ se imenuje nicelni
prostor matirke A. Oznacujemo ga z N(A).\\
\textit{neformalno: mnozica vektorjev, ki se z neko matriko zmnozijo v nicelni vektor. Matriko A samo eliminiras po gaussu in nato dobljene resitve enacis z 0.}

\textbf{3.8} Ce je matrika A kvadratna in ni obrnljiva, potem N(A) vsebuje samo vektor $\vec{0}$

\textbf{3.9} Matrika ima \textit{stopnicasto} obliko, kadar se vsaka od njenih vrstic zacne z vsaj eno
niclo vec kot prejsnja vrstica.

\textbf{3.10} Prvi element, razlicen od nic v vsaki vrstici, je \textit{pivot}. Stevilo pivotov v matriki
se imenuje rang matrike. Rang matrike A zapisemo kot $rang(A)$.

\textbf{3.11} Rang matrike ni vecji od stevila vrstic in ni vecji od stevila stolpcev matrike.

\textbf{3.12}
\begin{center}
    \textit{Stevilo prostih neznank matrike = st. stolpcev - rang matrike}   
\end{center}

\textbf{3.13}
\begin{enumerate}
    \item Visoka in ozka matrika $(m > n)$ ima poln stolpicni rang, kadar je $rang(A) = n$
    \item Nizka in siroka matrika $(m < n)$ ima poln vrsticni rang, kadar je $rang(A) = m$
    \item Kvadratna matrika $(n = m)$ ima poln rang, kadar je $rang(A) = m = n$
\end{enumerate}

\textbf{3.14} Za vsako matriko A s polnim stolpicnim rangom $r = n \leq m$, velja:
\begin{enumerate}
    \item Vsi stolpci A so pivotni stolpci
    \item Sistem enacb $A\vec{x} = \vec{0}$ nima prostih neznank, zato tudi nima posebnih resitev
    \item Nicelni prostor $N(A)$ vsebuje le nicelni vektor $N(A) = \{\vec{0}\}$
    \item Kadar ima sistem enacb $A\vec{x} = \vec{b}$ resitev(kar ni vedno res!), je resitev ena sama
    \item Reducirana vrsticna oblika matrike (A) se da zapisati kot
\end{enumerate}
\begin{center}
    $R =
    \begin{bmatrix}
        I\\
        0
    \end{bmatrix}
    \begin{bmatrix}
        n \times n\; enotska\; matrika\\
        m - n\; vrstic\; samih\; nicel\;
    \end{bmatrix}
    $
\end{center}

\textbf{3.15} Za vsako matriko A s polnim vrsticnim rangom $r = m \leq n$ velja:
\begin{enumerate}
    \item Vse vrstice so pivotne, ni prostih vrstic in U (stopnicasta oblika) in R(reducirana stopnicasta oblika) nimata nicelnih vrstic
    \item Sistem enacb $A\vec{x} = \vec{b}$ je resljiv za vsak vektor $\vec{b}$
    \item Sistem $A\vec{x} = \vec{b}$ ima $n-r = n-m$ prostih neznank, zato tudi prav toliko posebnih resitev
    \item Stolpicni prostor $C(A)$ je ves prostor $R^{m}$
\end{enumerate}

\textbf{3.16} Za vsako kvadratno matriko A polnega ranga (rang(A) = m = n) velja:
\begin{enumerate}
    \item Reducirana vrsticna oblika matrike A je enotska matrika
    \item Sistem enacb $A\vec{x} = \vec{b}$ ima natancno eno resitev za vsak vektor desnih strani $\vec{b}$
    \item Matrika A je obrnljiva
    \item Nicelni prostor matrike A je samo nicelni vektor $N(A) = \{\vec{0}\}$
    \item Stolpicni prostor matrike A je cel prostor $C(A) = R^{m}$
\end{enumerate}

\textbf{3.17} Vektorji $\vec{x_{1}}, \dots,\vec{x_{n}}$ so linearno neodvisni, ce je
\begin{center}
    $ 0\vec{x_{1}} + 0\vec{x_{2}} + \dots + 0\vec{x_{n}}$
\end{center}
edina njihova linearna kombinacija, ki je enaka vektorju $\vec{0}$. Vektorji $\vec{x_{1}}, \dots,\vec{x_{n}}$ so
linearno odvisni, \textit{ce niso linearno neodvisni}.

\textbf{3.18} Ce so vektorji \textit{odvisni}, lahko vsaj enega izrazimo z ostalimi.

\textbf{3.19} Ce je med vektorji  $\vec{u_{1}}, \dots,\vec{u_{n}}$ tudi nicelni vektor, so 
vektorji \textit{linearno odvisni}.

\textbf{3.20} Vsaka mnozica n vektorjev iz $R^{n}$ je odvisna, kadar je $n > m $.

\textbf{3.21} Stolpci matrike A so linearno neodvisni natanko tedaj, ko ima homogena enacba
$A\vec{x} = \vec{0}$ edino resitev $\vec{x} = \vec{0}$.

\textbf{3.22} Kadar je $rang(A) = n$, so stolpci matrike $A \in R^{m \times n}$ linearno
neodvisni. Kadar je pa $rang(A) < n$, so stolpci matrike $A \in R^{m \times n}$ linearno odvisni.

\textbf{3.23} Kadar je $rang(A) = m$, so vrstice matrike $A \in R^{m \times n}$ linearno neodvisne.
Kadar je pa $rang(A) < m$, so vrstice matrike $A \in R^{m \times n}$ linearno odvisne.

\textbf{3.24} Vrsticni prostor matrike A je podprostor v $R^{n}$, ki ga razpenjajo vrstice matrike A.

\textbf{3.25} Vrsticni prostor matrike A je $C(A^{T})$, stolpicni prostor matrike $A^{T}$.

\textbf{3.26} \textit{Baza vektorskega prostora} je mnozica vektorjev, ki
\begin{enumerate}
    \item je linearno neodvisna in
    \item napenja cel prostor.
\end{enumerate}

\textbf{3.27} Vsak vektor iz vektorskega prostora lahko na en sam nacin izrazimo
kot linearno kombinacijo baznih vektorjev.
 
\textbf{3.28} Vektorji $\vec{x_{1}}, \dots,\vec{x_{n}}$ so baza prostora $R^{n}$ natanko tedaj, kadar 
je matrika, sestavljena iz stolpcev $\vec{x_{1}}, \dots,\vec{x_{n}}$, obrnljiva.

\textbf{3.29} Prostor $R^{n}$ ima za $n > 0$ neskoncno mnogo razlicnih baz.

\textbf{3.30} Ce sta mnozici vekotrjev {$\vec{v_{1}}, \dots,\vec{v_{m}}$} in $\vec{u_{1}}, \dots,\vec{u_{n}}$
obe bazi istega vektorskega prostora, potem je $m = n \implies$ vse baze istega vektorskega prostora imajo
isto stevilo vektorjev.

\textbf{3.31} \textit{Dimenzija} vektroskega prostora je stevilo baznih vektorjev.

\textbf{3.32} Dimenziji stolpicnega prostora $C(A)$ in vrsticnega prostora $C(A^{T})$ sta enaki rangu matrike $A$
\begin{center}
    $dim(C(A)) = dim(C(A^{T})) = rang(A)$.
\end{center}

\textbf{3.33} Dimenzija nicelnega prostora $N(A)$ matrike A z $n$ stolpci in ranga $r$
je enaka $dim(N(A)) = n - r$.

\textbf{3.34} Stolpicni prostor $C(A)$ in vrsticni prostor $C(A^{T})$ imata oba dimenzijo r. Dimenzija
nicelnega prostora $N(A)$ je $n -r$, Dimenzija levega nicelnega prostora $N(A^{T})$ pa je $m - r$.

\textbf{3.35} Vsako matriko ranga 1 lahko zapisemo kot produkt(stolpcnega) vektorja z vrsticnim
vektorjem $A = \vec{u}\vec{v}^{T}$.

\section{\underline{Linearne preslikave}}

\textbf{4.1} Preslikava $A: U \rightarrow V$ je linearna, ce velja
\begin{enumerate}
    \item aditivnost: $A(\vec{u}_{1} + \vec{u}_{2}) = A\vec{u}_{1} + A\vec{u}_{2}$ za vse $\vec{u}_{1}, \vec{u}_{2} \in U$,
    \item homogenost: $A(\alpha \vec{u}) = \alpha(A\vec{u})$ za vse $\alpha \in R$ in $\vec{u} \in U$.
\end{enumerate}
\textbf{Pozor!} Preslikava ni linearna, ce $A(\vec{0}) \neq  \vec{0}$.

\textbf{4.2} Preslikava $A: U \rightarrow V$ je linearna natanko tedaj, ko velja
\begin{center}
    $A(\alpha_{1}\vec{u}_{1} + \alpha_{2}\vec{u}_{2}) = \alpha_{1}A\vec{u}_{1} + \alpha_{2}A\vec{u}_{2}$
\end{center}
za vse $\alpha_{1}, \alpha_{2} \in R$ in vse $\vec{u}_{1}, \vec{u}_{2} \in U$.

\textbf{4.3} Ce je A \textit{linearna preslikava}, je $A\vec{0} = \vec{0}$.

\textbf{4.4} Naj bo $A: U \rightarrow V$ linearna preslikava in $\sum_{i=1}^{k} \alpha_{i}\vec{u}_{i}$
linearna kombinacija vektorjev. Potem je A($\sum_{i=1}^{k} \alpha_{i}\vec{u}_{i}$) = $\sum_{i=1}^{k} \alpha_{i}A\vec{u}_{i}$.

\textbf{4.5} Naj bo $\beta =$ $\{ \vec{u_{1}}, \dots,\vec{u_{n}}\}$ baza za vektorski prostor U. Potem je linearna
preslikava $A: U \rightarrow V$ natanko dolocena, ce poznamo slike baznih vektorjev.

\textbf{4.6} Naj bo $\beta =$ $\{\vec{u_{1}}, \dots,\vec{u_{n}}\}$ baza za U in $\{\vec{v_{1}}, \dots,\vec{v_{n}}\}$.
Potem obstaja natanko ena linearna preslikava $A: U \rightarrow V$, za katero je $A\vec{u}_{i} = \vec{v}_{i}$ za $i = 1, 2, \dots, n$.

\textbf{4.7} Naj bo $A: U \rightarrow V$ linearna preslikava. Potem mnozico
\begin{center}
    $ker A = \{ \vec{u} \in U; A\vec{u} = \vec{0}\}$
\end{center}
imenujemo \textit{jedro} linearne preslikave. Ker je $A\vec{0} = \vec{0}$, je $\vec{0} \in$ ker A za vse A.
Zato je jedro vedno neprazna mnozica.
\textit{Ce je matrika A$\phi$ \textbf{enotska} preslikava za } $\phi$, \textit{potem velja}
\begin{center}
    \begin{math}
        ker \phi = N(A).
    \end{math}
\end{center}

\textbf{4.8} Jedro linearne preslikave $A: U \rightarrow V$ je vektorski podprostor v U.

\textbf{4.9} Mnozico
\begin{center}
    $im\; A = \{ \vec{v} \in V; obstaja\; tak\; \vec{u} \in U,\; da\; je\; \vec{v} = A\vec{u} \}$
\end{center}
imenujemo \textit{slika} linearne preslikave $A: U \rightarrow V$.
\textit{Ce je matrika A$\phi$ \textbf{enotska} preslikava za } $\phi$, \textit{potem velja}
\begin{center}
    \begin{math}
        im \phi = C(A).
    \end{math}
\end{center}

\textbf{4.10} Ce je $A: U \rightarrow V$ linearna preslikava, potem je njena slika $im\; A$
vektorski podprostor v V.

\textbf{4.11} Ce je $A: U \rightarrow V$ linearna preslikava, in je rang matrike te preslikave v standardni bazi poln,
potem lahko sklepamo, da ima  ta preslikava \textbf{trivialno jedro}.


\section{\underline{Ortogonalnost}}

\textbf{5.1} Podprostora $U$ in $V$ vektorskega prostora sta med seboj ortogonalna,
ce je vsak vektor $\vec{u} \in U$ ortogonalen na vsak vektor $\vec{v} \in V$.

\textbf{5.2} Za vsako matriko $A \in R^{m \times n}$ velja:
\begin{enumerate}
    \item Nicelni prostor $N(A)$ in vrsticni prostor $C(A^{T})$ sta ortogonalna podprostora $R^{n}$
    \item Levi nicelni prostor $N(A^{T})$ in stolpicni prostor $C(A)$ sta ortogonalna podprostora prostora $R^{m}$.
\end{enumerate}

\textbf{5.3} Ortogonalni komplement $V^{\perp}$ podprostora V vsebuje VSE vektorje, ki so ortogonalni na V.

\textbf{5.4} Naj bo A matrika dimenzije $m \times n$.
\begin{itemize}
    \item Nicelni prostor $N(A)$ je ortogonalni\\ komplement vrsticnega prostora $C(A^{T})$ v prostoru $R^{n}$
    \item Levi nicelni prostor $N(A^{T})$ je ortogonalni komplement stolpicnega prostora $C(A)$ v prostoru $R^{m}$.
\end{itemize}

\textbf{5.5} Za vsak vektor $\vec{y}$ v stolpicnem prostoru $C(A)$ obstaja v vrsticnem prostoru $C(A^{T})$ en sam
vektor $\vec{x}$, da je $A\vec{x} = \vec{y}$.

\textbf{5.6} Ce so stolpci matrike A linearno neodvisni, je matrika $A^{T}A$ obrnljiva.

\textbf{5.7} Matrika P je projekcijska, kadar
\begin{itemize}
    \item je simetricna: $P^{T} = P$ in
    \item velja $P^{2} = P$.
\end{itemize}

\textbf{5.8} Ce je P projekcijska matrika, ki projecira na podprostor U, potem je $I -P$ projekcijska
matrika, ki projecira na $U^{\perp}$, ortogonalni komplement podprostora U.

\textbf{5.9} Vektorji $\vec{q_{1}}, \vec{q_{2}}, \dots, \vec{q_{n}}$ so ortonormiranim kadar so ortogonalni in imanjo vsi
dolzino 1, torej
\begin{center}
    $\vec{q_{i}}^{T}\vec{q_{i}} = $ \Bigg\{ 
    $\begin{matrix}
        0\;  ko\; je\; i \neq j\; pravokotni\; vektorji\\
        1\;  ko\; je\; i = j\; enotski\; vektorji
    \end{matrix}$
\end{center}
za matriko $Q =$ [$\vec{q_{1}}, \vec{q_{2}} \dots \vec{q_{n}}$]  velja $Q^{T}Q = I$.

\textbf{5.10} Vektorji $\vec{q_{1}}, \dots, \vec{q_{n}}$ naj bodo ortonormirani v prostoru $R^{m}$. Potem
za matriko
\begin{center}
    $Q = \begin{bmatrix}
        \vec{q_{1}} \vec{q_{2}} \dots \vec{q_{n}}
    \end{bmatrix}$
\end{center}
velja, da je $Q^{T}Q = I_{n}$ enotska matrika reda n.

\textbf{5.11} Matrika Q je ortogonalna, kadar je
\begin{enumerate}
    \item kvadratna in
    \item ima ortonormirane stolpce.
\end{enumerate}

\textbf{5.12} Ce je Q ortogonalna matirka, potem je obrnljiva in $Q^{-1} = Q^{T}$.

\textbf{5.13} Mnozenje z ortogonalno matriko ohranja dolzino vektorjev in kote med njimi. Ce je Q
ortogonalna matrika, potem je 
\begin{center}
    $|| Q \vec{x} || = || \vec{x} ||$ za vsak vektor $\vec{x}$ in\\
    $(Q\vec{x})^{T}Q\vec{y} = \vec{x^{T}} \vec{y}$ za vsak vektor $\vec{x}$ in $\vec{y}$
\end{center}

\textbf{5.14} Ce sta $Q_{1}$ in $Q_{2}$ ortogonalni matriki, je tudi produkt $Q = Q_{1}Q_{2}$ ortogonalna
matrika.

\textbf{5.15} Iz linearno neodvisnih vektorjev $a_{1}, \dots, a_{n}$ z \textit{Gram-Schmidtovo} ortogonalizacijo
dobimo ortonormirane vektorje $q_{1}, \dots, q_{n}$. Matriki A in Q s temi stolpci zadoscajo enacbi $A = QR$, kjer
je R zgornjetrikotna matrika.

\textbf{5.16} Vektorski prostor $\iota$ je mnozica vseh neskoncnih zaporedij $\vec{u}$ s koncno
dolzino
\begin{center}
    $||\vec{u}||^{2} = \vec{u} \cdot \vec{u} = \vec{u_{1}}^{2} + \vec{u_{2}}^{2} + \dots < \infty$
\end{center}

\textbf{5.17} Polinomi $p_{0}(x), p_{1}(x), \dots, p_{n}(x), \dots$ sestavljajo zaporedje
ortogonalnih polinomov, kadar
\begin{enumerate}
    \item $p_{i}(x)$ je polinom stopnje i
    \item $(p_{i}(x), p_{j}(x)) = 0$, kadarkoli je $i \neq j$.
\end{enumerate}

\section{\underline{Determinante}}

\textbf{6.1} Determinanta enotske matirke je\\ $det(I) = 1$.
\begin{center}
    \begin{math}
        \begin{vmatrix}
            1 & 0\\
            0 & 1
        \end{vmatrix}
        = 1\; in\;
        \begin{vmatrix}
            1 & & 0\\
            & \ddots &\\
            0 & & 1\\
        \end{vmatrix}
        = 1.
    \end{math}
\end{center}

\textbf{6.2} Determinanta spremeni predznak, ce med seboj zamenjamo dve vrstici.

\textbf{6.3} Determinanta je linearna funkcija vsake vrstice posebej. To pomeni, da se
\begin{enumerate}
    \item determinanta pomnozi s faktorjem t, ce eno vrstico determinante(vsak element v tej vrstici)
    pomnozimo s faktorjem t.
    \begin{center}
        \begin{math}
            \begin{vmatrix}
                ta & tb\\
                c  & d\\
            \end{vmatrix}
            = t
            \begin{vmatrix}
                a & b\\
                c & d\\
            \end{vmatrix}
        \end{math}
    \end{center}
    \item determinanta je vsota dveh determinant, ki se razlikujeta le v eni vrstici,
    ce je v provitni determinanti ta vrstica vsota obeh vrstic, ostale vrstice pa so enake
    v vseh treh determinantah.
    \begin{center}
        \begin{math}
            \begin{vmatrix}
                a + a' & b + b'\\
                c      &      d\\
            \end{vmatrix} =
            \begin{vmatrix}
                a & b\\
                c & d\\
            \end{vmatrix} +
            \begin{vmatrix}
                a' & b'\\
                c & d\\
            \end{vmatrix}
        \end{math}
    \end{center}
\end{enumerate}

\textbf{Pozor!} Kadar mnozimo matriko A s skalarjem t, se vsak element matrike pomnozi s skalarjem.
Ko racunamo determinanto produkta matirke s skalarjem $tA$, skalar $t$ izpostavimo iz vsake vrstice posebej,
zato je $det(tA) = t^{n}det(A)$, kjer je $n$ stevilo vrstic (ali stolpcev) determinante.

\textbf{6.4} Matrika, ki ima dve enaki vrstici, ima determinanto enako 0.

\textbf{6.5} Ce v matriki od poljubne vrstice odstejemo mnogokratnik neke druge vrstice,
se njena determinanta ne spremeni.

\textbf{6.6} Naj bo $A$ poljubna kvadratna matirka $n \times n$ in $U$ njena vrsticno-stopnicasta
oblika, ki jo dobimo z \textit{Gaussovo eliminacijo}. Potem je 
\begin{center}
    $det(A) = \pm det(U)$.
\end{center}

\textbf{6.7} Determinanta, ki ima vrstico samih nicel, je enaka 0.

\textbf{6.8} Determinanta trikotne matrike $A$ je produkt diagonalnih elementov:
\begin{center}
    $det(A) = a_{11}a_{22} \hdots a_{nn}$.
\end{center}

\textbf{6.9} Determinanta singularne matrike je enaka 0, determinanta obrnljive matrike je razlicna od 0.

\textbf{6.10} Determinanta produkta dveh matrik je enaka produktu determinant obeh matrik:
\begin{center}
    $det(AB) = det(A)det(B)$.
\end{center}

\textbf{6.11} Determinanta inverzne matrike je enaka
\begin{center}
    $det(A^{-1}) = 1/det(A)$
\end{center}
in determinanta potence $A^{n}$ matrike A je
\begin{center}
    $det(A^{n}) = (det(A))^{n}$.
\end{center}

\textbf{6.12} Transponirana matrika $A^{T}$ ima isto determinanto kot A.

\textbf{6.13 Recap dovoljenih operacij nad determinanto}
\begin{enumerate}
    \item Ce zamenjamo dve vrstici, se \textbf{spremeni} predznak determinante
    \item Vrednost determinante se ne spremeni, ce neki vrstici pristejemo poljuben veckratnik katerekoli druge vrstice.
    \item Ce vse elemente neke vrstice pomnozimo z istim stevilom $\alpha$, se vrednost determinante pomnozi z $\alpha$.
\end{enumerate}

\textbf{6.14} Vsaka lastnost, ki velja za vrstice determinante, velja tudi
za njene \textbf{stolpce}. Med drugim:
\begin{itemize}
    \item Determinanta spremeni predznak, ce med seboj zamenjamo dva stolpca
    \item Determinanta je enaka 0, ce sta dva stolpca enaka
    \item Determinanta je enaka 0, ce so v vsaj enem stolpcu same nicle.
\end{itemize}

\textbf{6.15 (kofaktorska formula)} Ce je A kvadratna matrika reda n,
njeno determinanto lahko izracunamo z razvojem po $i-ti$ vrstici
\begin{center}
    $det(A) = a_{i1}C_{i1} + a_{i2}C_{i2} + \hdots + a_{in}C_{in}$.
\end{center}
Kofaktorje $C_{ij}$ izracunamo kot $C_{ij} = (-1)^{i+j}D_{ij}$, kjer je $D_{ij}$ determinanta,
ki jo dobimo, ce v A izbrisemo i-to vrstico in j-ti stolpec.

\textbf{6.16} Inverzna matrika $A^{-1}$ matrike A je transponirana matrika kofaktorjev,
deljena z determinanto $|A|$:
\begin{center}
    $A^{-1} = \frac{C^{T}}{det(A)}$,
\end{center}
kjer je C matrika kofaktorjev matrike A.

\textbf{6.17} Ploscina paralelograma, dolocenega z vektorjema $\vec{a}$ in $\vec{b} \in R^{2}$ je
enaka det([$\vec{a} \vec{b}$]), to je absolutni vrednosti determinante s stolpcema $\vec{a}$ in $\vec{b}$.

\textbf{6.18} Mesani produkt vektorjev $\vec{a}$ in $\vec{b}$ in $\vec{c}$ je enak determinanti matrike, ki 
ima te tri vektorje kot stolpce.

\textbf{6.19} Naj bo A matrika $R^{n\times n}$
\begin{center}
    \begin{math}
        A\; je\; obrnljiva\; \iff detA \neq 0
    \end{math}
\end{center}
\begin{center}
    \begin{math}
        A^{-1}\; ne\; obstaja\; \iff detA = 0
    \end{math}
\end{center}

\section{\underline{L. vrednosti in vektorji}}

\textbf{7.1} Vektor $\vec{x} \neq \vec{0}$, za katerega je $A\vec{x} = \lambda \vec{x}$ lastni vektor. Stevilo
$\lambda$ je lastna vrednost.
\textbf{Pozor!} Nicelni vektor $\vec{0}$ ne more biti lastni vektor. Lahko pa je lastna vrednost enaka 0.

\textbf{7.2} Ce ima matrika A lastno vrednost $\lambda$ in lastni vektor $\vec{x}$, potem ima matrika
$A^{2}$ lastno vrednost $\lambda^{2}$ in isti lastni vektor $\vec{x}$.

\textbf{7.3} Ce ima matrika A lastno vrednost $\lambda$ in lastni vektor $\vec{x}$, potem ima
matrika $A^{k}$ lastno vrednost $\lambda^{k}$ in isti lastni vektor $\vec{x}$.

\textbf{7.4} Ce ima matrika A lastno vrednost $\lambda$ in lastni vektor $\vec{x}$, potem ima
inverzna matrika lastno vrednost $1 / \lambda$ in isti lastni vektor $\vec{x}$.

\textbf{7.5} Sled kvadratne matrike A reda $n$ je vsota njenih diagonalnih elementov.
\begin{center}
    \begin{math}
        sled(A) =
        \sum_{i=1}^{n} a_{ii} =
        a_{11} + \dots + a_{nn}
    \end{math}.
\end{center}

\textbf{7.6} Sled matrike je enaka vsoti vseh lastnih vrednosti, stetih z njihovo veckratnostjo.
Ce so $\lambda_{1}, \dots, \lambda{n}$ lastne vrednosti matrike reda n, potem je sled enaka \textit{vsoti}
\begin{center}
    \begin{math}
        sled(A) =
        \sum_{i=1}^{n} \lambda_{i} =
        \lambda_{1} + \dots + \lambda_{n}
    \end{math},
\end{center}
determinanta matrike pa \textit{produktu} lastnih vrednosti
\begin{center}
    \begin{math}
        det(A) =
        \prod_{i=1}^{n} \lambda_{i} =
        \lambda_{1} \dots  \lambda_{n}
    \end{math}.
\end{center}

\textbf{7.7} Ce ima matrika A lastno vrednost $\lambda$, ki ji pripada lastni vektor $\vec{x}$,
potem ima matrika $A + cI$ lastno vrednost $\lambda + c$ z istim lastnim vektorjem $\vec{x}$ (velja samo z
enotskimi matrikami I).

\textbf{7.8} Lastne vrednosti trikotne matrike so enake diagonalnim elementom.

\textbf{7.9} Denimo, da ima matrika $A \in R^{n \times n}\; n$ linearno neodvisnih lastnih vektorjev
$\vec{x}_{1}, \vec{x}_{2}, \dots, \vec{x}_{n}$. Ce jih zlozimo kot stolpce v matriko S
\begin{center}
    \begin{math}
        S =
        \begin{bmatrix}
            \vec{x}_{1}, \vec{x}_{2}, \dots, \vec{x}_{n}
        \end{bmatrix}
    \end{math},
\end{center}
potem je T =: $S^{-1}AS$ diagonalna matrika z lastnimi vrednostmi $\lambda_{i}, i = 1, \dots, n$ na diagonali
\begin{center}
    \begin{math}
        S^{-1}AS = T =
        \begin{bmatrix}
            \lambda_{1} & &\\
            &   \ddots  &  \\
            & &     \lambda_{1} 
        \end{bmatrix}
    \end{math}.
\end{center}

\textbf{Pozor!} Lastni vektorji v matriki S morajo biti v istem vrstnem redu kot lastne vrednosti v matriki $T$.

\textbf{7.10} Ce je $A = STS^{-1}$, potem je $A^{k} = ST^{k}S^{-1}$ za vsak $k \in N$.

\textbf{7.11} Naj bo A kvadratna matrika reda n, ki ima n linearno neodvisnih vektorjev in $\vec{y}_{0} \in R^{n}$.
Zaporedje vektorjev iz $R^{n}$ naj bo definirano z $\vec{z}_{k+1} = A\vec{y}_{k}$. Potem velja
\begin{itemize}
    \item Ce je za vsaj eno lastno vrednost $|\lambda_{i}| > 1$, potem zaporedje
    $\vec{y}_{k}$ neomejeno narasca.
    \item Ce so vse lastne vrednosti $|\lambda_{i}| < 1$, potem zaporedje
    $\vec{y}_{k}$, konvergira proti nicelnemu vektorju $\vec{0}$.
    \item Ce je ena lastna vrednost $\lambda_{i} = 1$, vse ostale pa $|\lambda < 1|$, zaporedje
    $\vec{y}_{k}$ konvergira proti $c_{i}\vec{x}_{i}$.
\end{itemize}

\textbf{7.12} Vse lastne vrednosti realne simetricne matrike so realne.

\textbf{7.13} Lastni vektorji realne simetricne matrike, ki pripadajo razlicnim lastnim
vrednostim, so med seboj pravokotni.

\textbf{7.14 Schurov izrek} Za vsako kvadratno matriko reda n, ki ima le realne lastne vrednosti,
obstaja taka ortogonalna matrika $Q$, da je 
\begin{center}
    \begin{math}
        Q^{T}AQ = T
    \end{math}
\end{center} 
zgornjetrikotna matrika, ki ima lastne vrednosti(lahko so kompleksne) matrike A na diagonali.

\textbf{7.15 Spektralni izrek} Vsako simetricno matriko A lahko razcepimo v produkt
$A = QTQ^{T}$, kjer je Q ortogonalna matrika lastnih vektorjev, T pa diagonalna z lastnimi
vrednostmi matrike A na diagonali.

\textbf{7.16} Vsako realno simetricno matriko lahko zapisemo kot linearno kombinacijo matrik ranga 1
\begin{center}
    \begin{math}
        A = \lambda_{1}\vec{q}_{1}\vec{q}_{1}^{T} + \lambda_{2}\vec{q}_{2}\vec{q}_{2}^{T} +
        \dots + \lambda_{n}\vec{q}_{n}\vec{q}_{n}^{T} 
    \end{math},
\end{center}
kjer so $\vec{q}_{i}$ stolpci matrike Q (torej lastni vektorji matrike A).

\textbf{6.17} Za simetricno nesingularno matriko A je stevilo pozitivnih pivotov enako
stevilu pozitivnih lastnih vrednosti.

\textbf{6.18} Kvadratna matrika je pozitivno definirana, kadar so vse njene lastne vrednosti pozitivne.

\textbf{6.19} Kvadratna matrika reda 2 je pozitivno definirana natanko tedaj, kadar sta 
pozitivni sled in determinanta matrike.

\textbf{6.20} Simetricna matrika A reda $n$ je pozitivno definirana natanko tedaj, ko je za vsak
vektor $\vec{x} \neq \vec{0} \in R^{n}$
\begin{center}
    $\vec{x}^{T}A\vec{x} > 0$
\end{center}

\textbf{6.21} Ce sta matriki A in B pozitivno definitni, je pozitivno definitna tudi 
njuna vsota $A + B$.

\textbf{6.22} Matrika A je pozitivno definitna, kadar so vse njene vodilne glavne poddeterminante pozitivne.

\textbf{6.23} Ce so stolpci matrike R linearno neodvisni, je matrika $A = R^{T}R$ pozitivno definitna.

\textbf{6.24} Za vsako simetricno pozitivno definitno matriko A obstaja zgornjetrikotna matrika R, da
je $A = R^{T}R$.

\textbf{6.25} Simetricna matrka reda $n$, ki ima eno od spodnjih lastnosti, ima tudi ostale stiri:
\begin{enumerate}
    \item Vseh $n$ pivotov je pozitivnih;
    \item Vseh $n$ vodilnih glavnih determinant je pozitivnih;
    \item Vseh $n$ lastnih vrednosti je pozitivnih;
    \item Za vsak $\vec{x} \neq \vec{0}$ je $\vec{x}^{T}A\vec{x} > 0$;
    \item $A= R^{T}R$ za neko matriko R z linearno neodvisnimi stolpci.
\end{enumerate}

\textbf{6.26} Vsako realno $m \times n$ matriko A lahko zapisemo kot produkt
$A = UEV^{T}$, kjer je matrika U ortogonalna $m \times m$, E diagonalna $m \times n$ in
V ortogonalna $n \times n$.

\textbf{6.27} Ce je  matrika A simetricna in so vsej njeni elementi realni, potem je njen rang enak stevilu nenicelnih lastnih
vrednosti matrike A.
\begin{center}
    $rang(A)$ = stevilo $\lambda A$
\end{center}

\end{multicols}
\end{document}