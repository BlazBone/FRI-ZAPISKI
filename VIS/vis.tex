\documentclass{article}
\usepackage[margin=0.15cm]{geometry}
\usepackage{amsmath}
\usepackage{multicol}
\usepackage{hyperref}

\setlength{\columnseprule}{0.5pt}

\begin{document}

\begin{center}
    {\small VIS/FRI \par}
\end{center}

\begin{multicols}{3}

\textit{Ponovitev osnov}

\section{\underline{Kombinatorika}}


\textbf{1.1 Permutacije}
\begin{enumerate}
    \item brez ponavljanja: $P_{n} = n!$
    \item s ponavljanjem: $P_{n}^{k_{1},..,k_{n}} = \dfrac{n!}{k_{1}!,..,k_{n}!}$
\end{enumerate}

\textbf{1.2 Variacije}
\begin{enumerate}
    \item brez ponavljanja: $V_{n}^{r} = \dfrac{n!}{(n - r)!}$
    \item s ponavljanjem: $V_{n}^{r} = n^{r}$
\end{enumerate}

\textbf{1.3 Kombinacije}
\begin{enumerate}
    \item brez ponavljanja: ${n\choose k} = \dfrac{n!}{(n - k)! k!}$
    \item s ponavljanjem: $({n\choose k}) = {n + k - 1\choose k}$
\end{enumerate}
\begin{center}
    Lastnosti binomskega simbola:
    \begin{math}
        {n \choose n} = 1 \quad
        {n \choose 0} = 1 \quad
        {n \choose 1} = n \quad
        {n \choose r} = {n \choose n - r} 
    \end{math}
    Binomski izrek:\\
    \begin{math}
        (a + b)^{n} =
        {n \choose 0} a^{n} b^{0} + {n \choose 1} a^{n - 1} b^{1} +
        {n \choose 2} a^{n - 2} b^{2} + \dots + {n \choose n} a^{0} b^{n}
    \end{math}

\end{center}
Za kombinacije velja, da vrstni red \textbf{ni} pomemben. Medtem pa
ko v splosnem za variacije in permutacije velja, da vrstni red \textbf{je}
pomemben.

\section{\underline{Verjetnost}}


\textbf{2.1 Elementarna verjetnost}
\smallskip

Izid iz dane mnozice izidov je izbran na slepo, ce so vsi izidi
iz te mnozice enako verjetni. Takrat se dogodek $A$ zgodi z verjetnostjo:
\begin{center}
    \begin{math}
        P(A) = \dfrac{st.\: izidov,\: ki\: so\: v\: A\:}{st.\: vseh\: izidov}
    \end{math}    
\end{center}
Nasprotni dogodek pa z verjetnostjo:
\begin{center}
    \begin{math}
        P(\overline{A}) = 1 - P(A)
    \end{math}    
\end{center}
Nacelo vkljucitev in izkljucitev dogodkov:
\begin{center}
    \begin{math}
        P(A_{1} \cup \dots \cup A_{n}) =
    \end{math}
    \begin{math}
        P(A_{1}) + \dots + P(A_{n})
    \end{math}
    \begin{math}
        - P(A_{1} A_{2}) - P(A_{1} A_{3}) - \dots - P(A_{n - 1} A_{n})
    \end{math}
    \begin{math}
        + P(A_{1} A_{2} A_{3}) +P (A_{1} A_{2} A_{4}) + \dots + P(A_{n - 2} A_{n - 1} A_{n})
    \end{math}
    \begin{math}
        - \dots
    \end{math}
    \begin{math}
        + (-1)^{n + 1} P(A_{1} \dots A_{n})
    \end{math}
\end{center}
Dogodki $A_{1}, A_{2}, \dots , A_{k}$ in $B$ so \textbf{neodvisni}, ce velja
\begin{center}
    \begin{math}
        P(A_{1} \dots A_{k}) = P(A_{1}) \dots P(A_{k})
    \end{math}
\end{center}
\textit{ali z drugimi besedami...} Verjetnost produkta paroma neodvisnih
dogodkov je enaka produktu vrjetnosti teh dogodkov.

\textbf{2.2 Pogojna verjetnost}
Verjetnost da se zgodi dogodek A, ce vemo, da se zgodi dogodek B, je
\begin{center}
    \begin{math}
        P(A | B) = \dfrac{P(A \cap B)}{P(B)} = \dfrac{P(A)P(B|A)}{P(B)}
    \end{math}
\end{center}
Dogodka $A$ in b sta \textbf{neodvisna}, ce velja $P(A | B) = P(A)$ ali
$P(A B) = P(A)P(B)$ .
Pazi! Za par \textbf{nezdruzljivih} dogodkov $A$ in $B$
pa velja $P(AB) = 0$,  $P(A + B) = P(A) + P(B)$, $P(A|B) = 0$ in $P(B|A) = 0$.

\textbf{2.3 Popolna verjetnost}

Dogodki $H_{1}, H_{2}, \dots H_{n}$ tvorijo \textbf{popoln sistem dogodkov},
ce se nobena dva dogodka ne moreta zgoditi hrkati in se vedno
zgodi vsaj en od njih. Ce dogodki izpolnjujejo ta pogoj, potem po
nacelu vkljucitev/izkljucitev velja:
\begin{center}
    \begin{math}
        P(A) =
    \end{math}
    \smallskip
    \begin{math}
        \sum_{i=1}^{\infty} P(A \cap H_{i}) =
        \sum_{i=1}^{\infty} P(H_{1}) P(A | H_{i})
    \end{math}
\end{center}
% Za \textbf{popolni sistem dogodkov} velja unija hipotez:
% \begin{center}
%     \begin{math}
%         P(A|H_{1} \cup \dots \cup H_{n}) = 
%     \end{math}
%     \begin{math}
%         \dfrac{
%             P(A | H_{1}) P(H_{1}) + \dots
%             P(A | H_{n - 1}) P(H_{n - 1}) +
%             P(A | H_{n}) P(H_{n})
%         }
%         {
%             P(H_{1}) + \dots + P(H_{n - 1}) + P(H_{n})
%         }
%     \end{math}    
% \end{center}
Zanje velja tudi \textbf{Bayesova formula}:
\begin{center}
    \begin{math}
        P(H_{i} | A) = 
    \end{math}
    \begin{math}
        \dfrac{
            P(H_{i}) P(A | H_{i})
        }
        {
            P(A)
        }
    \end{math}
    \begin{math}
        = \dfrac{
            P(H_{i}) P(A | H_{i})
        }
        {
            \sum_{k=1}^{n} P(H_{k}) P(A | H_{k})
        }
    \end{math}        
\end{center}

\textbf{2.4 Geometrijska verjetnost}


Tocka je izbrana \textit{na slepo} iz intervala, lika, telesa.. ce za
vsak dogodek $A$ velja:
\begin{center}
    \begin{math}
        P(A) =
    \end{math}
    \smallskip
    \begin{math}
       \dfrac{
        mera\: izidov,\: ki\: so\: v\: A\:
        }
       {mera\: vseh\: izidov\:}
    \end{math}
\end{center}
Pri tem je mera lahko dolzina, ploscina, volumen,..
Basically upas da narises graf pravilno.

Splosno za vse nastete verjetnosti velja:
\begin{center}
    \begin{math}
    P(A \cup B) = P(A) + P(B) - P(A \cap B)
    \end{math} in 
    \begin{math}
        P(A \cap B) = P(A | B) P(B) = P(B | A) P(A)
    \end{math}
\end{center}

\section{\underline{Dss in porazdelitve}}

\textbf{3.1 Diskretna slucjana spremenljivka}
Naj bo $X$ diskretna slucajna spremenljivka $\implies$ $X$ je funkcija
s koncno ali stevno zalogo  vrednosti ${ a_{1}, a_{2}, \dots}$ Verjetnost, 
da $X$ zavzame vrednost $a_{i} \in R$, oznacimo z $P(X = a_{i}) = p_{i}$. Porazdelitev
$X$ lahko podamo na dva enakovredna nacina, in sicer s:
\begin{enumerate}
    \item s \textbf{porazdelitveno shemo}
        \begin{center}
            \begin{math}
                X \sim
                \begin{pmatrix}
                    a_{1} & a_{2} & a_{3} & \dots \\
                    p_{1} & p_{2} & p_{3} & \dots
                \end{pmatrix}
            \end{math}
        \end{center}
        velja $0 \leq p_{i} \leq 1$ in $p_{1} + p_{2} + \dots = 1$
    \item s \textbf{porazdelitveno funkcijo}
        \begin{center}
            \begin{math}
                F_{x}(x) := P(X \leq x)
            \end{math}
        \end{center}
\end{enumerate}

\textbf{3.2 Bernoullijeva} slucajna spremenljivka
\begin{center}
    \begin{math}
        X \sim B(p)
    \end{math}
\end{center}
\begin{itemize}
    \item V vsakem poskusu ima dogodek $A$ verjetnost $p$, $X$ pa ima vrednost 1, ce se je zgodil dogodek A, in 0 sicer.
    \item $P(X = 1) = p, P(X = 0) = 1 - p$
\end{itemize}


\textbf{3.3 Binomska} slucajna spremenljivka
\begin{center}
    \begin{math}
        X \sim B(n, p)
    \end{math}
\end{center}
\begin{itemize}
    \item $X$ je stevilo pojavitev izida $A$ v $n$ ponovitvah poskusa
    \item $P(X = k) = {n \choose k} p^{k} (1 - p)^{(n - k)}$ za $k = 0,1, \dots, n.$
\end{itemize}
Izvajamo $n$ neodvisnih slucajnih poskusov. V vsakem poskusu se lahko zgodi dogodek $A$ s 
konstantno verjetnostjo $p$, $p =  P(A)$. 
$X$ nam pove kolikokrat se je zgodil dogodek $A$ v $n$ poskusih.
npr. kovanec vrzemo 10x, koliksne so vrjetnosti, da pade cifra 0x, 2x, vsaj 3x,..
ali 5x vrzemo posteno kocko, izracunaj stevilo sestic, ki pade $\implies B(5, \dfrac{1}{6})$

\textbf{3.4 Geometrijska} slucajna spremenljivka
\begin{center}
    \begin{math}
        X \sim G(p)
    \end{math}
\end{center}
\begin{itemize}
    \item $X$ je stevilo ponovitev poskusa do (vkljucno) prve ponovitve izida $A$.
    \item $P(X = k) = (1 - p)^{k - 1} p$ za $k = 1,2, \dots$ 
    \item $P(X \leq k) = 1 - (1 - p)^{k}$ za $k = 1,2, \dots$
\end{itemize}
Izvajamo  neodvisne slucajne poskuse, dokler se ne zgodi dogodek $A$. V vsakem poskusu
se lahko zgodi dogodek $A$  s \textbf{konstantno} verjetnostjo $p$, $p =  P(A)$.
npr. koliko metov kocke je potrebnih, do prve sestice $\implies G(1/6)$.

\textbf{3.5 Pascalova} oz. \textbf{negativna binomska} slucajna spremenljivka
\begin{center}
    \begin{math}
        X \sim P(n, p)
    \end{math}
\end{center}
\begin{itemize}
    \item $X$ je stevilo ponovitev poskusa do (vkljucno) $n$-te ponovitve izida $A$.
    \item \begin{math}
        P(X = k) = {k - 1 \choose n - 1 }(1 - p)^{k - n} p^{n} $ za $ k = n, n + 1, n + 2, \dots
    \end{math}
\end{itemize}
npr. koliko metov kocke je potrebnih, dokler sestica ne pade 5x $\implies P(5, \dfrac{1}{6})$. Stevilo metov kovanca,
dokler grb ne pade 2x $\implies P(2, \dfrac{1}{2})$.

\textbf{3.6 Hipergeometrijska} slucjana spremenljivka
\begin{center}
    \begin{math}
        X \sim H(K, N - K, n)
    \end{math}
\end{center}
\begin{itemize}
    \item $X$ je stevilo elementov z doloceno lastnostjo med izbranimi.
    \item \begin{math}
        P(X = k) =  \dfrac{ {K \choose k} {N - K \choose n - k} }{{N \choose n}} $ za k = $ 0,1,2,\dots min\{ n, R \}
    \end{math}
\end{itemize}
V populaciji $N$  imamo $K$ elementov  z doloceno lastnostjo. Izbiramo brez vracanja $n$ elementov.
npr. koliko pikov med 7 kartami, ki smo jih na slepo izbrali izmed 16 kart, kjer so bli stirje piki.
imamo 400 ljudi, 100 brezposlenih, nakljucno jih izberemo 10. Zanima nas kaksna verjetnost je da sta 
2 izmed teh brezposelna $\implies P(x=2) = H(100, 400-100, 10)$. 

\textbf{3.7 Poissonova} slucajna spremenljivka
\begin{center}
    \begin{math}
        X \sim P(\lambda)
    \end{math}
\end{center}
\begin{itemize}
    \item $X$ je stevilo ponovitev dogodka A na danem intervalu, pri cemer:
        \begin{itemize}
            \item se dogodki pojavljajo neodvisno
            \item povprecno stevilo dogodgov $\lambda$, ki se pojavjio na dolocenem intervalu, je konstantno.
        \end{itemize}
    \item \begin{math}
        P(X = k) =  \dfrac{ \lambda^{k} }{k!} e^{- \lambda}$ za k = $ 0,1,2,\dots
    \end{math}
\end{itemize}
npr. ce se dogodek pojavi v povprecju 3x na minuto, lahko uporabimo poissa za izracun
kolikokrat se bo dogodek zgodil v  1/4h $\implies P(45)$. St avtomobilov, ki preckajo cesto v 1min.

\section{\underline{Zss in porazdelitve}}

\textbf{3.1 Zvezna slucjana spremenljivka}
Naj bo $X$ zvezna slucajna spremenljivka $\implies$ $X$ je realna funkcija,
za katero obstaja integrabilna funkcija $p_{X}: R \rightarrow [0, \infty)$,
tako da za vsak $x \in R$ velja:

\begin{center}
    \begin{math}
        F_{X}(x) := P(X \leq x) = \int_{- \infty}^{x} p_{X}(t) \,dt
    \end{math}
\end{center}

Funkciji $p_{X}$ pravimo \textbf{gostota verjetnosti}, funkciji $F_{X}$ pa
\textbf{porazdelitvena} funkcija. Mnozici vrednosti, ki jih zavzame spremenljivka
$X$, pravimo \textbf{zaloga vrednosti} in jo oznacimo z $Z_{X}$.
Lastnosti:
\begin{itemize}
    \item \begin{math}
        \int_{- \infty}^{+ \infty} p_{X}(x) \,dx = 1
    \end{math}
    \item \begin{math}
        P(a < X < b) = \int_{a}^{b} p_{X}(x) \,dx = F_{X}(b) - F_{X}(a),\: a,b \in R,\: a < b
    \end{math}
    \item \begin{math}
        P(X = a) = 0, a \in R
    \end{math} nqot
\end{itemize}
\textbf{ce} je funkcija zvezna v $x$, potem za njo velja tudi $F'(x) = p(x)$.

\textbf{3.2 Enakomerna zvezna} slucajna spremenljivka
\begin{center}
    \begin{math}
        X \sim U[a, b]
    \end{math}
\end{center}

\begin{itemize}
    \item  \begin{math}
        p_{X}(x) =
        \Bigg \{\begin{tabular}{ccc}
          $\frac{1}{b-a}$  & $x \in [a, b]$ & \\
          $0$ & $sicer$ & \\
        \end{tabular}
    \end{math} 
    
     \item \begin{math}
        F_{X}(x) =
        \Bigg \{\begin{tabular}{ccc}
          $0$ & $x < a$ & \\
          $\frac{x - a}{b - a}$  & $x \in [a, b]$ & \\
          $1$ & $x > b$  & \\
        \end{tabular}
    \end{math}
\end{itemize}

Vsi izidi na intervalu $[a,\: b]$ so enako verjetni.

\textbf{3.3 Eksponentna} slucajna spremenljivka
\begin{center}
    \begin{math}
        X \sim \epsilon(\lambda)
    \end{math}
\end{center}

\begin{itemize}
    \item  \begin{math}
        p_{X}(x) =
        \Bigg \{\begin{tabular}{ccc}
          $0$  & $x < 0$ & \\
          $\lambda e^{- \lambda x}$ & $x \geq 0$ & \\
        \end{tabular}
    \end{math} 
    
     \item \begin{math}
        F_{X}(x) =
        \Bigg \{\begin{tabular}{ccc}
          $0$ & $x < 0$ & \\
          $1 - e^{- \lambda x}$ & $x > 0$  & \\
        \end{tabular}
    \end{math}
\end{itemize}

Slucajna spremenljivka $X$ - cas med zaporednima dogodkoma,
pri cemer so dogodki neodvisni in se pojavijo s konstantno
stopnjo $\lambda$. $\lambda$ predstavlja povprecno stevilo dogodkov
na izbrano casovno enoto.

\textbf{3.4 Normalna} slucajna spremenljivka
\begin{center}
    \begin{math}
        X \sim N(\mu, \sigma )
    \end{math}
\end{center}

\begin{itemize}
    \item  \begin{math}
        p_{X}(x) = \frac{1}{\sigma \sqrt{2 \pi}} e^{- \frac{(x - \mu)^{2}}{2 \sigma^{2}}}
    \end{math} za $x \in R$ 
    
     \item Za $F_{X}(x)$ ne obstaja eksplicitna formula. Vrednost preberemo iz porazdelitvenih tabel.
\end{itemize}
Po centralnem limitnem izreku sta vsota in povprecje veliko neodvisnih, enako porazdeljenih
spremenljivk, \textit{normalno porazdeljeni}.
Porzadelitev $N(0, 1)$ je standardna normalna porazdelitev $\implies$ potem za vsak $x$ velja
\begin{math}
    P(X < x) = 1 - P(X > x)
\end{math}.

\textbf{3.5 Gamma} slucajna spremenljivka
\begin{center}
    \begin{math}
        X \sim \Gamma(n, \lambda)
    \end{math}
\end{center}

\begin{itemize}
    \item  \begin{math}
        p_{X}(x) =
        \Bigg\{\begin{tabular}{ccc}
          $0$  & $x \leq 0$ & \\
          $\frac{\lambda^{n} x^{n- 1} e^{ - \lambda x}}{\Gamma(n)}$ & $x > 0$ & \\
        \end{tabular}
    \end{math} 
\end{itemize}

V povprecju imamo na casovno enoto $\lambda$ ponovitev dogodka $A$, $X$ pa je cas med
prvo in $(n + 1)$ ponovitvijo dogodka $A$.

\textbf{3.5 Hi kvadrat} slucajna spremenljivka
\begin{center}
    \begin{math}
        X \sim  \chi^{2}(n) = \Gamma(\frac{n}{2}, \frac{1}{2})
    \end{math}
\end{center}

\begin{itemize}
    \item  \begin{math}
        p_{X}(x) =
        \Bigg\{\begin{tabular}{ccc}
          $0$  & $x \leq 0$ & \\
          $\dfrac{x^{\frac{n}{2} - 1} e ^{- \frac{x}{2}}}{2^{\frac{n}{2}} \Gamma(\dfrac{n}{2})}$ & $x > 0$ & \\
        \end{tabular}
    \end{math} 
\end{itemize}

Je vsota kvadratov $n$ neodvisnih standardnih normalnih slucajnih spremenljivk.
\bigskip

\textbf{appx. Odvodi}
\begin{center}
    \begin{enumerate}
        \item \begin{math}
            \frac{1}{x} = -\frac{1}{x^2}
        \end{math}
        \item \begin{math}
            x^n  = nx^{n-1}
        \end{math}
        \item \begin{math}
            \sqrt{x} = \frac{1}{2 \sqrt{x}}
        \end{math}
        \item \begin{math}
            \sqrt[n] x = \frac{1}{n \sqrt[n]{x^{n-1}}}
        \end{math}
        \item \begin{math}
            \sin (a x) =  a  \cos a x
        \end{math}
        \item  \begin{math}
            \cos (a x) = - a \sin (a x)
        \end{math}
        \item \begin{math}
            \tan x = \frac{1}{\cos^2 x} 
        \end{math}
        \item \begin{math}
            e^ax = ae^{ax}
        \end{math}
        \item \begin{math}
            a^x = a^x \ln a
        \end{math}
        \item \begin{math}
            x^x = x^x (1+\ln x)
        \end{math}
        \item \begin{math}
            ln x = \frac{1}{x}
        \end{math}
        \item \begin{math}
            log_a x = \frac{1}{x \ln a}
        \end{math}
        \item \begin{math}
            \arcsin x = \frac{1}{\sqrt {1 - x^2}}
        \end{math}
        \item \begin{math}
            \arccos x = - \frac{1}{\sqrt{1 - x^2}}
        \end{math}
        \item \begin{math}
            \arctan x = \frac{1}{1 + x^2}
        \end{math}
        \item \begin{math}
            \operatorname{arccot}x = -\frac{1}{1 + x^2}
        \end{math}
    \end{enumerate}
\end{center}
\textbf{appx. Integrali}
\begin{center}
    \begin{enumerate}
        \item \begin{math}
            \int x^a\,dx =
            \Bigg\{\begin{tabular}{ccc}
                $\frac{x^{a+1}}{a+1} + C$  & $a \neq -1$ & \\
                $ \ln{\left|x\right|} + C$ & $a = -1$ & \\
              \end{tabular}
        \end{math}
        \item \begin{math}
            \int \ln {x}\,dx = x \ln {x} - x + C
        \end{math}
        \item \begin{math}
            \int \frac {1}{\sqrt{x}}\,dx=2\sqrt{x} + C 
        \end{math}
        \item \begin{math}
            \int e^x\,dx = e^x + C
        \end{math}
        \item \begin{math}
            \int a^x\,dx = \frac{a^x}{\ln{a}} + C
        \end{math}
        \item \begin{math}
            \int \cos({ax}) \, dx = {sin (ax) \over a} + C
        \end{math}
        \item \begin{math}
            \int \sin({ax}) \, dx = {-{cos(ax) \over a}} + C
        \end{math}
        \item \begin{math}
            \int \tan{x} \, dx = -\ln{\left| \cos {x} \right|} + C
        \end{math}
        \item \begin{math}
            \int \frac{dx}{\cos^2 x}=\int \sec^2 x \, dx = \tan x + C
        \end{math}
        \item \begin{math}
            \int \frac{dx}{\sin^2 x}=\int \csc^2 x \, dx = -\cot x + C
        \end{math}
        \item \begin{math}
            \int {1 \over \sqrt{1-x^2}} \, dx = \arcsin {x} + C
        \end{math}
    \end{enumerate}
\end{center}
\smallskip
\end{multicols}
\end{document}