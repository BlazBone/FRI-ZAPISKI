\documentclass{article}
\usepackage[margin=0.15cm]{geometry}
\usepackage{amsmath}
\usepackage{multicol}
\usepackage{hyperref}

\setlength{\columnseprule}{0.5pt}

\begin{document}

\begin{center}
    {\small VIS/FRI \par}
\end{center}

\begin{multicols}{3}

\textit{Ponovitev analize}\\
\textbf{Odvodi}
\begin{center}
    \begin{small}
        \begin{enumerate}
            \item \begin{math}
                \frac{1}{x} = -\frac{1}{x^2}
            \end{math}
            \item \begin{math}
                x^n  = nx^{n-1}
            \end{math}
            \item \begin{math}
                \sqrt{x} = \frac{1}{2 \sqrt{x}}
            \end{math}
            \item \begin{math}
                \sqrt[n] x = \frac{1}{n \sqrt[n]{x^{n-1}}}
            \end{math}
            \item \begin{math}
                \sin (a x) =  a  \cos a x
            \end{math}
            \item  \begin{math}
                \cos (a x) = - a \sin (a x)
            \end{math}
            \item \begin{math}
                \tan x = \frac{1}{\cos^2 x} 
            \end{math}
            \item \begin{math}
                e^ax = ae^{ax}
            \end{math}
            \item \begin{math}
                a^x = a^x \ln a
            \end{math}
            \item \begin{math}
                x^x = x^x (1+\ln x)
            \end{math}
            \item \begin{math}
                ln x = \frac{1}{x}
            \end{math}
            \item \begin{math}
                log_a x = \frac{1}{x \ln a}
            \end{math}
            \item \begin{math}
                \arcsin x = \frac{1}{\sqrt {1 - x^2}}
            \end{math}
            \item \begin{math}
                \arccos x = - \frac{1}{\sqrt{1 - x^2}}
            \end{math}
            \item \begin{math}
                \arctan x = \frac{1}{1 + x^2}
            \end{math}
            \item \begin{math}
                \operatorname{arccot}x = -\frac{1}{1 + x^2}
            \end{math}
        \end{enumerate}
    \end{small}
\end{center}
\textbf{Integrali}
\begin{center}
    \begin{small}
        \begin{enumerate}
            \item \begin{math}
                \int x^a\,dx =
                \Bigg\{\begin{tabular}{ccc}
                    $\frac{x^{a+1}}{a+1} + C$  & $a \neq -1$ & \\
                    $ \ln{\left|x\right|} + C$ & $a = -1$ & \\
                  \end{tabular}
            \end{math}
            \item \begin{math}
                \int \ln {x}\,dx = x \ln {x} - x + C
            \end{math}
            \item \begin{math}
                \int \frac {1}{\sqrt{x}}\,dx=2\sqrt{x} + C 
            \end{math}
            \item \begin{math}
                \int e^x\,dx = e^x + C
            \end{math}
            \item \begin{math}
                \int a^x\,dx = \frac{a^x}{\ln{a}} + C
            \end{math}
            \item \begin{math}
                \int \cos({ax}) \, dx = { \frac{sin (ax)}{a} } + C
            \end{math}
            \item \begin{math}
                \int \sin({ax}) \, dx = { \frac{-cos(ax)}{a} } + C
            \end{math}
            \item \begin{math}
                \int \tan{x} \, dx = -\ln{\left| \cos {x} \right|} + C
            \end{math}
            \item \begin{math}
                \int \frac{dx}{\cos^2 x}=\int \sec^2 x \, dx = \tan x + C
            \end{math}
            \item \begin{math}
                \int \frac{dx}{\sin^2 x}=\int \csc^2 x \, dx = -\cot x + C
            \end{math}
            \item \begin{math}
                \int {\frac{1}{\sqrt{1-x^2}}} \, dx = \arcsin {x} + C
            \end{math}
        \end{enumerate}
    \end{small}
\end{center}
    \textbf{Integriranje absolutnih vrednosti} (primer):
    \begin{small}
        Imamo funkcijo $f(x) = |x|$ , ki je zvezna na intervalu $[-1, 1]$
        Ce hocemo to funkcijo integrirati in zelimo izracunati njeno 
        \textit{porazdelitveno} funkcijo integrirati locimo 2 primera:    
    \end{small}
    \begin{enumerate}
        \item \begin{math}
            -1 \leq x < 0\\
            F(x) = \int_{-1}^x |t|\,dt = \int_{-1}^x -t\,dt = - \frac{t^2}{2} \rvert_{-1}^{x} = - \frac{1}{2} (x^2 -1)
        \end{math}
        \item \begin{math}
            0 \leq x < 1 \\
            F(x) = \int_{-1}^x |t|\,dt = \int_{-1}^0 -t\,dt + \int_{0}^x t\,dt = - \frac{t^2}{2} \rvert_{0}^{-1} + - \frac{t^2}{2} \rvert_{0}^{x}= \frac{1}{2} (1 + x^2)
        \end{math}
    \end{enumerate}
% other useful formulas
\begin{small}
    \begin{math}
        \sqrt[n] x = (x)^{\frac{1}{n}}
    \end{math}    
\end{small}


\section{\underline{Kombinatorika}}


\textbf{1.1 Permutacije}
\begin{small}
    \begin{enumerate}
        \item brez ponavljanja: $P_{n} = n!$
        \item s ponavljanjem: $P_{n}^{k_{1},..,k_{n}} = \dfrac{n!}{k_{1}!,..,k_{n}!}$
    \end{enumerate}        
\end{small}

\textbf{1.2 Variacije}
    \begin{small}
        \begin{enumerate}
            \item brez ponavljanja: $V_{n}^{r} = \dfrac{n!}{(n - r)!}$
            \item s ponavljanjem: $V_{n}^{r} = n^{r}$
        \end{enumerate}                
    \end{small}

\textbf{1.3 Kombinacije}
\begin{small}
    \begin{enumerate}
        \item brez ponavljanja: ${n\choose k} = \dfrac{n!}{(n - k)! k!}$
        \item s ponavljanjem: $({n\choose k}) = {n + k - 1\choose k}$
    \end{enumerate}
    \begin{center}
        Lastnosti binomskega simbola:
        \begin{math}
            {n \choose n} = 1 \quad
            {n \choose 0} = 1 \quad
            {n \choose 1} = n \quad
            {n \choose r} = {n \choose n - r} 
        \end{math}
        Binomski izrek:\\
        \begin{math}
            (a + b)^{n} =
            {n \choose 0} a^{n} b^{0} + {n \choose 1} a^{n - 1} b^{1} +
            {n \choose 2} a^{n - 2} b^{2} + \dots + {n \choose n} a^{0} b^{n}
        \end{math}
    
    \end{center}        

Za kombinacije velja, da vrstni red \textbf{ni} pomemben. Medtem pa
ko v splosnem za variacije in permutacije velja, da vrstni red \textbf{je}
pomemben.
\end{small}


\section{\underline{Verjetnost}}


\textbf{2.1 Elementarna verjetnost}
\smallskip

Izid iz dane mnozice izidov je izbran na slepo, ce so vsi izidi
iz te mnozice enako verjetni. Takrat se dogodek $A$ zgodi z verjetnostjo:
\begin{center}
    \begin{math}
        P(A) = \dfrac{st.\: izidov,\: ki\: so\: v\: A\:}{st.\: vseh\: izidov}
    \end{math}    
\end{center}
Nasprotni dogodek pa z verjetnostjo:
\begin{center}
    \begin{math}
        P(\overline{A}) = 1 - P(A)
    \end{math}    
\end{center}
Nacelo vkljucitev in izkljucitev dogodkov:
\begin{center}
    \begin{small}
        \begin{math}
            P(A_{1} \cup \dots \cup A_{n}) =
        \end{math}
        \begin{math}
            P(A_{1}) + \dots + P(A_{n})
        \end{math}
        \begin{math}
            - P(A_{1} A_{2}) - P(A_{1} A_{3}) - \dots - P(A_{n - 1} A_{n})
        \end{math}
        \begin{math}
            + P(A_{1} A_{2} A_{3}) +P (A_{1} A_{2} A_{4}) + \dots + P(A_{n - 2} A_{n - 1} A_{n})
        \end{math}
        \begin{math}
            - \dots
        \end{math}
        \begin{math}
            + (-1)^{n + 1} P(A_{1} \dots A_{n})
        \end{math}
    \end{small}
\end{center}
Dogodki $A_{1}, A_{2}, \dots , A_{k}$ in $B$ so \textbf{neodvisni}, ce velja
\begin{center}
    \begin{math}
        P(A_{1} \dots A_{k}) = P(A_{1}) \dots P(A_{k})
    \end{math}
\end{center}
\textit{ali z drugimi besedami...} Verjetnost produkta paroma neodvisnih
dogodkov je enaka produktu vrjetnosti teh dogodkov.

\textbf{2.2 Pogojna verjetnost}
Verjetnost da se zgodi dogodek A, ce vemo, da se zgodi dogodek B, je
\begin{center}
    \begin{math}
        P(A | B) = \dfrac{P(A \cap B)}{P(B)} = \dfrac{P(A)P(B|A)}{P(B)}
    \end{math}
\end{center}
Dogodka $A$ in b sta \textbf{neodvisna}, ce velja $P(A | B) = P(A)$ ali
$P(A B) = P(A)P(B)$ .
Pazi! Za par \textbf{nezdruzljivih} dogodkov $A$ in $B$
pa velja $P(AB) = 0$,  $P(A + B) = P(A) + P(B)$, $P(A|B) = 0$ in $P(B|A) = 0$.

\textbf{2.3 Popolna verjetnost}

Dogodki $H_{1}, H_{2}, \dots H_{n}$ tvorijo \textbf{popoln sistem dogodkov},
ce se nobena dva dogodka ne moreta zgoditi hrkati in se vedno
zgodi vsaj en od njih. Ce dogodki izpolnjujejo ta pogoj, potem po
nacelu vkljucitev/izkljucitev velja:
\begin{center}
    \begin{math}
        P(A) =
    \end{math}
    \smallskip
    \begin{math}
        \sum_{i=1}^{\infty} P(A \cap H_{i}) =
        \sum_{i=1}^{\infty} P(H_{1}) P(A | H_{i})
    \end{math}
\end{center}
Za \textbf{popolni sistem dogodkov} velja unija hipotez:
\begin{center}
    \begin{math}
        P(A|H_{1} \cup \dots \cup H_{n}) = 
    \end{math}
    \bigskip
    \begin{math}
        \frac{
            P(A | H_{1}) P(H_{1}) + \dots +
            P(A | H_{n}) P(H_{n})
        }
        {
            P(H_{1}) + \dots + P(H_{n - 1}) + P(H_{n})
        }
    \end{math}    
\end{center}
Zanje velja tudi \textbf{Bayesova formula}:
\begin{center}
    \begin{math}
        P(H_{i} | A) = 
    \end{math}
    \begin{math}
        \frac{
            P(H_{i}) P(A | H_{i})
        }
        {
            P(A)
        }
    \end{math}
    \begin{math}
        = \frac{
            P(H_{i}) P(A | H_{i})
        }
        {
            \sum_{k=1}^{n} P(H_{k}) P(A | H_{k})
        }
    \end{math}        
\end{center}

\textbf{2.4 Geometrijska verjetnost}


Tocka je izbrana \textit{na slepo} iz intervala, lika, telesa.. ce za
vsak dogodek $A$ velja:
\begin{center}
    \begin{math}
        P(A) =
    \end{math}
    \smallskip
    \begin{math}
       \dfrac{
        mera\: izidov,\: ki\: so\: v\: A\:
        }
       {mera\: vseh\: izidov\:}
    \end{math}
\end{center}
Pri tem je mera lahko dolzina, ploscina, volumen,..
Basically upas da narises graf pravilno.

Splosno za vse nastete verjetnosti velja:
\begin{center}
    \begin{math}
    P(A \cup B) = P(A) + P(B) - P(A \cap B)
    \end{math} in 
    \begin{math}
        P(A \cap B) = P(A | B) P(B) = P(B | A) P(A)
    \end{math}
\end{center}

\section{\underline{Diskretne s.s.}}

\textbf{3.1 Diskretna slucjana spremenljivka}
Naj bo $X$ diskretna slucajna spremenljivka $\implies$ $X$ je funkcija
s koncno ali stevno zalogo  vrednosti ${ a_{1}, a_{2}, \dots}$ Verjetnost, 
da $X$ zavzame vrednost $a_{i} \in R$, oznacimo z $P(X = a_{i}) = p_{i}$. Porazdelitev
$X$ lahko podamo na dva enakovredna nacina, in sicer s:
\begin{enumerate}
    \item s \textbf{porazdelitveno shemo}
        \begin{center}
            \begin{math}
                X \sim
                \begin{pmatrix}
                    a_{1} & a_{2} & a_{3} & \dots \\
                    p_{1} & p_{2} & p_{3} & \dots
                \end{pmatrix}
            \end{math}
        \end{center}
        velja $0 \leq p_{i} \leq 1$ in $p_{1} + p_{2} + \dots = 1$
    \item s \textbf{porazdelitveno funkcijo}
        \begin{center}
            \begin{math}
                F_{x}(x) := P(X \leq x)
            \end{math}
        \end{center}
\end{enumerate}

\textbf{3.2 Bernoullijeva} slucajna spremenljivka
\begin{center}
    \begin{math}
        X \sim B(p)
    \end{math}
\end{center}
\begin{itemize}
    \item V vsakem poskusu ima dogodek $A$ verjetnost $p$, $X$ pa ima vrednost 1, ce se je zgodil dogodek A, in 0 sicer.
    \item $P(X = 1) = p, P(X = 0) = 1 - p$
\end{itemize}


\textbf{3.3 Binomska} slucajna spremenljivka
\begin{center}
    \begin{math}
        X \sim B(n, p)
    \end{math}
\end{center}
\begin{itemize}
    \item $X$ je stevilo pojavitev izida $A$ v $n$ ponovitvah poskusa
    \item $P(X = k) = {n \choose k} p^{k} (1 - p)^{(n - k)}$ za $k = 0,1, \dots, n.$
\end{itemize}
Izvajamo $n$ neodvisnih slucajnih poskusov. V vsakem poskusu se lahko zgodi dogodek $A$ s 
konstantno verjetnostjo $p$, $p =  P(A)$. 
$X$ nam pove kolikokrat se je zgodil dogodek $A$ v $n$ poskusih.
npr. kovanec vrzemo 10x, koliksne so vrjetnosti, da pade cifra 0x, 2x, vsaj 3x,..
ali 5x vrzemo posteno kocko, izracunaj stevilo sestic, ki pade $\implies B(5, \dfrac{1}{6})$

\textbf{3.4 Geometrijska} slucajna spremenljivka
\begin{center}
    \begin{math}
        X \sim G(p)
    \end{math}
\end{center}
\begin{itemize}
    \item $X$ je stevilo ponovitev poskusa do (vkljucno) prve ponovitve izida $A$.
    \item $P(X = k) = (1 - p)^{k - 1} p$ za $k = 1,2, \dots$ 
    \item $P(X \leq k) = 1 - (1 - p)^{k}$ za $k = 1,2, \dots$
\end{itemize}
Izvajamo  neodvisne slucajne poskuse, dokler se ne zgodi dogodek $A$. V vsakem poskusu
se lahko zgodi dogodek $A$  s \textbf{konstantno} verjetnostjo $p$, $p =  P(A)$.
npr. koliko metov kocke je potrebnih, do prve sestice $\implies G(1/6)$.

\textbf{3.5 Pascalova} oz. \textbf{negativna binomska} slucajna spremenljivka
\begin{center}
    \begin{math}
        X \sim P(n, p)
    \end{math}
\end{center}
\begin{itemize}
    \item $X$ je stevilo ponovitev poskusa do (vkljucno) $n$-te ponovitve izida $A$.
    \item \begin{math}
        P(X = k) = {k - 1 \choose n - 1 }(1 - p)^{k - n} p^{n} $ za $ k = n, n + 1, n + 2, \dots
    \end{math}
\end{itemize}
npr. koliko metov kocke je potrebnih, dokler sestica ne pade 5x $\implies P(5, \dfrac{1}{6})$. Stevilo metov kovanca,
dokler grb ne pade 2x $\implies P(2, \dfrac{1}{2})$.

\textbf{3.6 Hipergeometrijska} slucjana spremenljivka
\begin{center}
    \begin{math}
        X \sim H(K, N - K, n)
    \end{math}
\end{center}
\begin{itemize}
    \item $X$ je stevilo elementov z doloceno lastnostjo med izbranimi.
    \item \begin{math}
        P(X = k) =  \dfrac{ {K \choose k} {N - K \choose n - k} }{{N \choose n}} $ za k = $ 0,1,2,\dots min\{ n, K \}
    \end{math}
\end{itemize}
V populaciji $N$  imamo $K$ elementov  z doloceno lastnostjo. Izbiramo brez vracanja $n$ elementov.
npr. koliko pikov med 7 kartami, ki smo jih na slepo izbrali izmed 16 kart, kjer so bli stirje piki.
imamo 400 ljudi, 100 brezposlenih, nakljucno jih izberemo 10. Zanima nas kaksna verjetnost je da sta 
2 izmed teh brezposelna $\implies P(x=2) = H(100, 400-100, 10)$.
\textbf{Pozor!} Na kolokviju/izpitu moras nujno zapisati tudi mozne vrednosti k-ja.

\textbf{3.7 Poissonova} slucajna spremenljivka
\begin{center}
    \begin{math}
        X \sim P(\lambda)
    \end{math}
\end{center}
\begin{itemize}
    \item $X$ je stevilo ponovitev dogodka A na danem intervalu, pri cemer:
        \begin{itemize}
            \item se dogodki pojavljajo neodvisno
            \item povprecno stevilo dogodgov $\lambda$, ki se pojavjio na dolocenem intervalu, je konstantno.
        \end{itemize}
    \item \begin{math}
        P(X = k) =  \dfrac{ \lambda^{k} }{k!} e^{- \lambda}$ za k = $ 0,1,2,\dots
    \end{math}
\end{itemize}
npr. ce se dogodek pojavi v povprecju 3x na minuto, lahko uporabimo poissa za izracun
kolikokrat se bo dogodek zgodil v  1/4h $\implies P(45)$. St avtomobilov, ki preckajo cesto v 1min.

\section{\underline{Zvezne s.s.}}

\textbf{4.1 Zvezna slucajna spremenljivka}
Naj bo $X$ zvezna slucajna spremenljivka $\implies$ $X$ je realna funkcija,
za katero obstaja integrabilna funkcija $p_{X}: R \rightarrow [0, \infty)$,
tako da za vsak $x \in R$ velja:

\begin{center}
    \begin{math}
        F_{X}(x) := P(X \leq x) = \int_{- \infty}^{x} p_{X}(t) \,dt
    \end{math}
\end{center}

Funkciji $p_{X}$ pravimo \textbf{gostota verjetnosti}, funkciji $F_{X}$ pa
\textbf{porazdelitvena} funkcija. Mnozici vrednosti, ki jih zavzame spremenljivka
$X$, pravimo \textbf{zaloga vrednosti} in jo oznacimo z $Z_{X}$.
Lastnosti:
\begin{itemize}
    \item \begin{math}
        \int_{- \infty}^{+ \infty} p_{X}(x) \,dx = 1
    \end{math}
    \item \begin{math}
        P(a < X < b) = \int_{a}^{b} p_{X}(x) \,dx = F_{X}(b) - F_{X}(a),\: a,b \in R,\: a < b
    \end{math}
    \item \begin{math}
        P(X = a) = 0, a \in R
    \end{math}
    \item \begin{math}
        P(|X| < 1) = P(-1 < X < 1)
    \end{math}
\end{itemize}
\textbf{ce} je funkcija zvezna v $x$, potem za njo velja tudi $F'(x) = p(x)$.
Za zvezno slucajno spremenljivko $X$ je \textit{funkcija prezivetja} $S(x) = P(X > x)$
vedno zvezna, nenarascujoca in zavzema vrednosti na intervalu $[0, 1]$.
\textbf{4.2 Enakomerna zvezna} slucajna spremenljivka
\begin{center}
    \begin{math}
        X \sim U[a, b]
    \end{math}
\end{center}

\begin{itemize}
    \item  \begin{math}
        p_{X}(x) =
        \Bigg \{\begin{tabular}{ccc}
          $\frac{1}{b-a}$  & $x \in [a, b]$ & \\
          $0$ & $sicer$ & \\
        \end{tabular}
    \end{math} 
    
     \item \begin{math}
        F_{X}(x) =
        \Bigg \{\begin{tabular}{ccc}
          $0$ & $x < a$ & \\
          $\frac{x - a}{b - a}$  & $x \in [a, b]$ & \\
          $1$ & $x > b$  & \\
        \end{tabular}
    \end{math}
\end{itemize}

Vsi izidi na intervalu $[a,\: b]$ so enako verjetni.

\textbf{4.3 Eksponentna} slucajna spremenljivka
\begin{center}
    \begin{math}
        X \sim \epsilon(\lambda)
    \end{math}
\end{center}

\begin{itemize}
    \item  \begin{math}
        p_{X}(x) =
        \Bigg \{\begin{tabular}{ccc}
          $0$  & $x < 0$ & \\
          $\lambda e^{- \lambda x}$ & $x \geq 0$ & \\
        \end{tabular}
    \end{math} 
    
     \item \begin{math}
        F_{X}(x) =
        \Bigg \{\begin{tabular}{ccc}
          $0$ & $x < 0$ & \\
          $1 - e^{- \lambda x}$ & $x \geq 0$  & \\
        \end{tabular}
    \end{math}
\end{itemize}

Slucajna spremenljivka $X$ - cas med zaporednima dogodkoma,
pri cemer so dogodki neodvisni in se pojavijo s konstantno
stopnjo $\lambda$. $\lambda$ predstavlja povprecno stevilo dogodkov
na izbrano casovno enoto.

\textbf{4.4 Normalna} slucajna spremenljivka
\begin{center}
    \begin{math}
        X \sim N(\mu, \sigma )
    \end{math}
\end{center}

\begin{itemize}
    \item  \begin{math}
        p_{X}(x) = \frac{1}{\sigma \sqrt{2 \pi}} e^{- \frac{(x - \mu)^{2}}{2 \sigma^{2}}}
    \end{math} za $x \in R$ 
    
     \item Za $F_{X}(x)$ ne obstaja eksplicitna formula. Vrednost preberemo iz porazdelitvenih tabel.
\end{itemize}
Po centralnem limitnem izreku sta vsota in povprecje veliko neodvisnih, enako porazdeljenih
spremenljivk, \textit{normalno porazdeljeni}.
Porzadelitev $N(0, 1)$ je standardna normalna porazdelitev $\implies$ potem za vsak $x$ velja
\begin{math}
    P(X < x) = 1 - P(X > x)
\end{math}.

\textbf{4.5 Gamma} slucajna spremenljivka
\begin{center}
    \begin{math}
        X \sim \Gamma(n, \lambda)
    \end{math}
\end{center}

\begin{itemize}
    \item  \begin{math}
        p_{X}(x) =
        \Bigg\{\begin{tabular}{ccc}
          $0$  & $x \leq 0$ & \\
          $\frac{\lambda^{n} x^{n- 1} e^{ - \lambda x}}{\Gamma(n)}$ & $x > 0$ & \\
        \end{tabular}
    \end{math} 
\end{itemize}

V povprecju imamo na casovno enoto $\lambda$ ponovitev dogodka $A$, $X$ pa je cas med
prvo in $(n + 1)$ ponovitvijo dogodka $A$.

\textbf{4.6 Hi kvadrat} slucajna spremenljivka
\begin{center}
    \begin{math}
        X \sim \chi^{2}(n) = \Gamma(\frac{n}{2}, \frac{1}{2})
    \end{math}
\end{center}

\begin{itemize}
    \item  \begin{math}
        p_{X}(x) =
        \Bigg\{\begin{tabular}{ccc}
          $0$  & $x \leq 0$ & \\
          $\dfrac{x^{\frac{n}{2} - 1} e ^{- \frac{x}{2}}}{2^{\frac{n}{2}} \Gamma(\frac{n}{2})}$ & $x > 0$ & \\
        \end{tabular}
    \end{math} 
\end{itemize}

Je vsota kvadratov $n$ neodvisnih standardnih normalnih slucajnih spremenljivk.

\section{\underline{Matematicno upanje}}

\textbf{5.1 Matematicno upanje} diskretne slucajne spremenljivke
\begin{center}
    \begin{small}
        \begin{math}
            X \sim
            \begin{pmatrix}
                a_{1} & a_{2} & a_{3} & \dots \\
                p_{1} & p_{2} & p_{3} & \dots \\
            \end{pmatrix}
        \end{math}
    \end{small}
\end{center}
oz. zvezne slucajne spremenljivke z gostoto $p_X$ je
\begin{center}
    \begin{math}
        E(X^n) = \sum_{k=0}^{\infty} x_k^n p_k\;
    \end{math} oz.\\
    \begin{math}
        E(X^n) = \int_{- \infty}^{ \infty} \textbf{x}^n p_X(x)\, dx
    \end{math}.
\end{center}
Za vsaki slucajni spremenljivki $X$ in $Y$ (lahko sta odvisni, lahko je ena zvezna in druga diskretna)
ter $a, b, n \in R$ velja 
\begin{center}
    \begin{math}
        E(aX + bY) = aE(X) + bE(Y)
    \end{math}
\end{center} in
\begin{center}
    \begin{math}
        nE(X^a Y^b) = nE(X^a) E(Y^b)
    \end{math}
\end{center} in
\begin{center}
    \begin{math}
        E((X + Y)^2) = E(X^2 + 2XY + Y^2)
    \end{math} itn...
\end{center}
Matematicno upanje nam pove pricakovano vrednost, kolikokrat oz. kdaj (odvisno od porazdelitve) 
se bo dolocen dogodek zgodil. Po definiciji disperzije velja tudi:
\begin{center}
    $E(X^2) = D(X) + E(X)^2$    
\end{center}

\textbf{5.2 Matematicno upanje funkcije}\\
$f:\; R \rightarrow R$ slucajne spremenljivke $X$ je
\begin{center}
    \begin{math}
        E(f(X)) = \sum_{k=0}^{\infty} f(x_k) p_k\;
    \end{math} oz.\\
    \begin{math}
        E(f(X)) = \int_{- \infty}^{ \infty} f(x) p_X (x)\, dx
    \end{math}.
\end{center}

\textbf{5.3 Matematicna upanja} dss in zss
\begin{itemize}
    \item \begin{math}
        X \sim Bernoulli(p) \implies E(X) = p
    \end{math}
    \item \begin{math}
        X \sim Binom(n, p) \implies E(X) = np
    \end{math}
    \item \begin{math}
        X \sim G(p) \implies E(X) = \frac{1}{p}
    \end{math}
    \item \begin{math}
        X \sim Pascal(n, p) \implies E(X) = \frac{n}{p}
    \end{math}
    \item
    \begin{math}
        X \sim H(R, B, n) \implies E(X) = \frac{nR}{R + B}
    \end{math}
    \item \begin{math}
        X \sim Pois(\lambda) \implies E(X) = \lambda
    \end{math}
    \item \begin{math}
        X \sim U[a, b] \implies E(X) = \frac{a + b}{2}
    \end{math}
    \item  \begin{math}
        X \sim \epsilon(\lambda) \implies E(X) = \frac{1}{\lambda}
    \end{math}
    \item \begin{math}
        X \sim N(\mu, \sigma ) \implies E(X) = \mu
    \end{math}
    \item \begin{math}
        X \sim \chi^{2}(n) \implies E(X) = n
    \end{math}
\end{itemize}

\section{\underline{Disperzija in std. odklon}}

\textbf{6.1 Disperzija} ali \textit{varianca} slucajnje spremenljivke $X$
je definirana kot
\begin{center}
    \begin{math}
        D(X) = E((X - E(X))^2) = E(X^2) - E(X)^2
    \end{math}
\end{center}
Za $a, b \in R$ velja
\begin{center}
    \begin{math}
        D(aX + b) = a^2 D(X)
    \end{math}.
\end{center} 
Ce sta $X$ in $Y$ \textit{neodvisni} je 
\begin{center}
    \begin{small}
        \begin{math}
            D(aX + bY) = a^2D(X) + b^2D(Y)
        \end{math}.
    \end{small}
\end{center}

\textbf{6.2 Standardni odklon} slucajnje spremenljivke $X$
je enak 
\begin{center}
    \begin{math}
        \sigma(X) = \sqrt{D(X)}
    \end{math}.
\end{center}

\textbf{6.3 Disperzije} dss in zss
\begin{itemize}
    \item \begin{math}
        X \sim B(p) \implies D(X) = p(1 - p)
    \end{math}
    \item \begin{math}
        X \sim B(n, p) \implies D(X) = np(1 - p)
    \end{math}
    \item \begin{math}
        X \sim G(p) \implies D(X) = \frac{1 - p}{p^2}
    \end{math}
    \item \begin{math}
        X \sim P(n, p) \implies D(X) = \frac{n(1 - p)}{p^2}
    \end{math}
    \item
    \begin{math}
        X \sim H(R, B, n) \implies
    \end{math}
    \begin{center}
        \begin{math}
            \frac{
                nRB(R + B - n)
            }{
                (R + B)^2 (R + B - 1)
            }
        \end{math}
    \end{center}
    \item \begin{math}
        X \sim P(\lambda) \implies D(x) = \lambda
    \end{math}
    \item \begin{math}
        X \sim U[a, b] \implies D(X) = \frac{(b - a)^2}{12}
    \end{math}
    \item  \begin{math}
        X \sim \epsilon(\lambda) \implies D(X) = \frac{1}{\lambda^2}
    \end{math}
    \item \begin{math}
        X \sim N(\mu, \sigma ) \implies D(X) = \sigma^2
    \end{math}
    \item \begin{math}
        X \sim \chi^{2}(n) \implies D(X) = 2n
    \end{math}
\end{itemize}

\section{\underline{Slucajni vektorji}}

\textbf{7.1} Diskretni slcuajni spremenljivki $X$ in $Y$  lahko
dolocata (dvorazsesni) \textbf{diskretni slucajni vektor} (X, Y).
Verjetnost, da (X, Y) zavzame vrednost $(x_i, y_i) \in R$,

\begin{center}
    oznacimo s $P(X = x_i, Y = y_j) = p_{ij}$.
\end{center}

Porazdelitev (X, Y) lahko podamo na dva enakovredna nacina, in sicer:

\begin{enumerate}
    \item s \textbf{porazdelitveno tabelo}
        \begin{center}
            \begin{tiny}
                (X, Y)
                \begin{tabular}{ |c|c|c|c|c|c|c| } 
                    \hline
                        $ X \symbol{92} Y $ & $y_1$ & $y_2$ & $\dots$ & $y_m$ & $\dots$ & $X$ \\
                        \cline{1-7}
                        $x_1$ & $p_{11}$ & $p_{12}$ & $\dots$ & $p_{1m}$ & $\dots$ & $p_1$ \\
                        $x_2$ & $p_{21}$ & $p_{22}$ & $\dots$ & $p_{2m}$ & $\dots$ & $p_2$ \\
                        $\vdots$ & $\vdots$ & $\vdots$ & $\vdots$ & $\vdots $ & $\vdots$ & $\vdots$ \\
                        $x_n$ & $p_{n1}$ & $p_{n2}$ & $\dots$ & $p_{nm}$ & $\dots$ & $p_n$ \\
                        $\vdots$ & $\vdots$ & $\vdots$ & $\vdots$ & $\vdots $ & $\vdots$ & $\vdots$ \\
                        \cline{1-7}
                        $ Y $ & $q_1$ & $q_2$ & $\dots$ & $q_m$ & $\dots$ & $1$ \\
                    \hline
                    \end{tabular}
            \end{tiny}
        \end{center}
        pri cemer je $0 \leq p_{ij} \leq 1$,
        \begin{math}
            \sum_{i = 1}^{\infty} \sum_{j = 1}^{\infty} p_{ij} = 1,\:
            \sum_{i = 1}^{\infty} p_{ij} = p_i
        \end{math} za vsak $i \in N$ in
        \begin{math}
            \sum_{i = 1}^{\infty} p_{ij} = q_j
        \end{math} za vsak $j \in N$.
    \item s \textbf{porazdelitveno funkcijo}
        \begin{center}
            \begin{math}
                F_{X,Y}(x, y) = P(X \leq x, Y \leq y).
            \end{math}
            
        \end{center}
        Velja
        \begin{math}
            F_{X,Y}(x, y) =
        \end{math}\\
        \begin{math}
            \sum_{i = 1}^{\infty} \sum_{j = 1}^{\infty} p_{ij}  I_{[x_i, \infty)} (x) I_{[y_j, \infty)} (y)
        \end{math}
        , kjer je
        \begin{center}
            \begin{math}
                I_{[x_i, \infty)} (x) = 
                \Bigg\{\begin{tabular}{cc}
                    $1$  & $x_i \leq x$ \\
                    $0$  & $sicer$ \\
                  \end{tabular}
            \end{math} \\
            \begin{math}
                I_{[y_j, \infty)} (y) = 
                \Bigg\{\begin{tabular}{cc}
                    $1$  & $y_j \leq y$ \\
                    $0$  & $sicer$ \\
                  \end{tabular}
            \end{math}
        \end{center}
\end{enumerate}
Podan imamo vektor $(X \in [0, a], Y \in [0,b])$. Potem velja slednje:
\begin{itemize}
    \item $P(X < 1 ) = P(X \leq 1, Y \leq b)$
    \item $P(X < 1, Y > \frac{1}{2}) = P(X \leq 1, Y \leq 1) - P(X \leq 1, Y \leq \frac{1}{2})$
    \item \begin{math}
        P(X > 1, Y > \frac{1}{2}) = 
    \end{math}
\end{itemize}
\begin{center}
    \begin{small}
        \begin{math}
            P(X \leq a, \frac{1}{2} \leq Y \leq b) - P(X \leq 1, \frac{1}{2} \leq Y \leq b) =
            (P(X \leq a, Y \leq b) - P(X \leq a, Y \leq \frac{1}{2})) - (P(X \leq 1, Y \leq b) - P(X \leq 1, Y \leq \frac{1}{2}))
        \end{math}
    \end{small}
\end{center}

\textbf{Robne porazdelitve} so porazdelitve komponent\\
\begin{center}
    \begin{math}
        p_i = P(X = x_i) = \sum_{i = 1}^{\infty} p_{ij}
    \end{math}\\
    \begin{math}
        q_j = P(Y = y_i) = \sum_{j = 1}^{\infty} p_{ij}
    \end{math}
\end{center}
Slucajni spremenljivki X in Y sta \textbf{neodvisni},
ce za poljublni stevili $x,y \in R$ velja
\begin{center}
    \begin{math}
        P(X = x, Y = y) = P(X = x)P(Y = y)
    \end{math} in
\end{center}
\begin{center}
    \begin{math}
        P(X = x\, |\, Y = y) = \frac{P(X = x, Y = y)}{P(Y = y)}
    \end{math}
\end{center}

\textbf{7.2 dvorazsezna gostota verjetnosti}
Naj bosta X,Y z.s.s. Par (X, Y) je \textit{zvezni slucajni vektor},
ce obstaja integrabilna funkcija $p_{X,Y}: R^2 \rightarrow R$ (gostota verjetnosti), tako da za
vsak par $(x,y) \in R^2$ velja
\begin{center}
    \begin{math}
        F_{X,Y}(x, y) = P(X \leq x, Y \leq y) = \int_{-\infty}^{x}\int_{-\infty}^{y} p_{X,Y}(x,y)\,dx\,dy
    \end{math}.
\end{center}
\begin{small}
Funkciji $F_{X,Y}$ pravimo \textit{porazdelitvena funkcija}.    
\end{small}
Velja
\begin{center}
    \begin{math}
        \int_{-\infty}^{\infty}\int_{-\infty}^{\infty} p_{X,Y}(x,y)\,dx\,dy = 1
    \end{math}
\end{center}
\textbf{Robni gostoti} sta
\begin{center}
    \begin{math}
        p_X(x) = \int_{-\infty}^{\infty} p_{XY}(x,y)\, dy
    \end{math} in
    \begin{math}
        p_Y(y) = \int_{-\infty}^{\infty} p_{XY}(x,y)\, dx
    \end{math}
\end{center}
Zvezni slucajni spremenljivki $X$ in $Y$ sta \textbf{neodvisni}, ce
za vsaki realni stevili $x, y \in R$ velja
\begin{center}
    \begin{math}
        p_{X,Y}(x,y) = p_X(x)p_Y(y)
    \end{math}.
\end{center}

\textbf{7.3 Matematicno upanje} funkcije\\
$f: R^2 \rightarrow R$ dvorazseznega slucajnega vektorja (X, Y) je 
za diskretni slucajni vektor definirano s predpisom
\begin{center}
    \begin{math}
        E(f(X,Y)) = 
    \end{math}\\
    \begin{math}
        \sum_{i=1}^{\infty}\sum_{j=1}^{\infty} f(x_i, y_i) P(X = x_i, Y = y_i)
    \end{math}
\end{center}
za zvezni slucjani vektor pa s predpisom
\begin{center}
    \begin{math}
        E(f(X,Y)) = 
    \end{math}\\
    \begin{math}
        \int_{-\infty}^{\infty}\int_{-\infty}^{\infty} f(x, y) P_{X,Y}(x,y)\,dx\,dy
    \end{math}.
\end{center}
Ce sta $X$ in $Y$ \textbf{neodvisni} velja
\begin{center}
    \begin{math}
        E(XY) = E(X)E(Y)
    \end{math}.
\end{center}

\textbf{7.4 Kovarianca} slucajnih spremenljivk $X$ in $Y$ je definirana kot
\begin{center}
    \begin{math}
        Cov(X,Y) = E((X - E(X)) (Y - E(Y))) = E(\underline{XY}) - E(X)E(Y)
    \end{math}.
\end{center}
\textbf{XY} mores posebi zracunat porazdelitev(ampak pazi, ni nujno da sta neodvisni, zato, ce imas tabelo, poberi vrednosti za npr. $P(XY = 1)$ iz tabele)!\\
Za disperzijo velja
\begin{center}
    \begin{math}
        D(X) = Cov(X, X)
    \end{math} in\\
    \begin{math}
        D(aX + bY) = a^2D(X) + b^2D(Y) + 2abCov(X, Y)
    \end{math}
\end{center}
Za slucajne spremenljivke $X, Y, Z$ ter $a,b \in R$ velja:
\begin{itemize}
    \item \begin{math}
        Cov(X + a, Y + b) = Cov(X, Y)
    \end{math},
    \item \begin{math}
        Cov(aX + bY, Z) =\\ aCov(X, Z) + bCov(Y,Z)
    \end{math},
    \item \begin{math}
        Cov(X, Y) = Cov(Y, X)
    \end{math},
    \item \begin{math}
        |Cov(X, Y)| \leq \sqrt{D(X)D(Y)}
    \end{math},
    \item \begin{math}
        Cov(aX, bY) = abCov(X, Y)
    \end{math}
    \item \begin{math}
        Cov(a, X) = 0 
    \end{math} \textit{(neodvisni)}
\end{itemize}

Ce sta s.s $X$ in $Y$ \textit{neodvisni} je njuna kovarianca enaka \textbf{0},
$Cov(X,Y) = 0$.

\textbf{7.5 Korelacijski koeficient} izracunamo po formuli
\begin{center}
    \begin{math}
        \rho(X,Y) = \frac{Cov(X,Y)}
                      {\sigma(X) \sigma(Y)}
    \end{math}
\end{center}
Korelacijski koeficient zavzema vrednosti na intervalu [-1, 1].
Ce velja $\rho(X, Y) = 0$, lahko sklepamo da sta spremenljivki $X$ in $Y$ \textbf{nekorelirani}.
Za $a, b, c, d \in R$ ter $a, c > 0$ velja:
\begin{center}
    $\rho(aX + b, cY + d) = \rho(X, Y)$
\end{center}
Ce je $|Cov(X, Y)| = \sqrt{D(X)D(Y)}$, tj. $\rho(X, Y) = \pm 1$, potem sta $X$ in $Y$ v \textbf{linearni zvezi}
\begin{center}
    \begin{math}
        Y = \pm \frac{D(Y)}{D(X)}(X - E(X)) + E(Y)
    \end{math}.
\end{center}
Ker iz neodvisnosti sledi E(XY) = E(X)E(Y), sta neodvisni slucajni spremenljivki tudi nekorelirani. Obratno
pa ne velja!

\section{\underline{Normalna porazdelitev}}

\textbf{8.1} Normalna porazdelitev je odvisna od dveh parametrov:
$\mu = E(X)\, in\, \sigma = \sigma(X)$. Gostota njene porazdelitve je:
\begin{center}
    \begin{math}
        p_{X}(x) = \frac{1}{\sigma \sqrt{2 \pi}} e^{- \frac{(x - \mu)^{2}}{2 \sigma^{2}}}
    \end{math}        
\end{center}
Standardizacija: 
\begin{center}
    \begin{math}
        Z = \frac{X - \mu}{\sigma} \sim N(\mu =  0, \sigma = 1)
    \end{math}
\end{center}
Vrednost $F(X) = P(X \leq x)$ dobimo tako da integriramo
funkcijo gostote na intervalu $[-\infty, x]$:
\begin{center}
    \begin{math}
        F(x) = \frac{1}{\sigma \sqrt{2 \pi}} \int_{-\infty}^x e^{-t^2 / 2}\, dt
    \end{math}.
\end{center}
Velja : $P (a \leq Z \leq b) = F(b) - F(a)$ in $x \geq 4 \Rightarrow F(x) \approx 1 (std.\, napaka)$.\\
Ce je spremenljivka $X \sim N(0,1)$ normalno porazdeljena, velja tudi da so lihi momenti normalne porazdelitve enaki 0 ($E(X^3) = E(X^5) = 0$).

\textbf{8.2 $\sigma$}  pravila

\begin{itemize}
    \item \begin{math}
        1 \sigma \Rightarrow P(\mu - \sigma \leq X \leq \mu + \sigma) = 0.683
    \end{math}

    \item \begin{math}
        2 \sigma \Rightarrow P(\mu - 2\sigma \leq X \leq \mu + 2\sigma) = 0.954
    \end{math}

    \item \begin{math}
        3 \sigma \Rightarrow P(\mu - 3\sigma \leq X \leq \mu + 3\sigma) = 0.997
    \end{math}
\end{itemize}

\textbf{8.3 $q_x$} pravila

\begin{itemize}
    \item $P (X \leq q_1) = 0.25$
    \item $P (X \leq q_2(m)) = 0.5$
    \item $P (X \leq q_3) = 0.75$
\end{itemize}

\textbf{8.4 Standardizacija binomske} porazdelitve
\begin{center}
    \begin{math}
        X_B \sim B(n, p)
    \end{math}
\end{center}
kjer velja $\mu = np$ in $\sigma = \sqrt{np(1-p)}$.
velja:
\begin{center}
    \begin{math}
        P(a \leq X_B \leq b) \approx P(a - \frac{1}{2} \leq X_N \leq b + \frac{1}{2})
        = F(\frac{b + \frac{1}{2} - \mu}{\sigma}) - F(\frac{a - \frac{1}{2} - \mu}{\sigma})
    \end{math}
\end{center}
in
\begin{center}
    \begin{math}
        P(X_B \leq b) \approx P(X_N \leq b + \frac{1}{2}) = F(\frac{b + \frac{1}{2} - \mu}{\sigma})
    \end{math}
\end{center}.
\textbf{Pazi} za normalizirane \textit{diskretne} porazdelitve velja:
\begin{itemize}
    \item $P(X_B < k) = P(X_B \leq k - 1)$
    \item $P(X_B > k) = P(X_B \geq k + 1) = 1 - P(X_B \leq k)$
\end{itemize}


\textbf{8.5 Aproksimacija binomske} porazdelitve\\

\textbf{8.5.1 Poissonov Priblizek}

Naj bo 
\begin{center}
    \begin{math}
        X_B \sim B(n, p)\: in\: X_P \sim P(np)
    \end{math}
\end{center}
Ce je 
\begin{itemize}
    \item $n \geq 20$ in $p \in (0, 0.05)$ ali pa
    \item $n \geq 100$ in $np \in (0, 10]$,
\end{itemize}
ponavadi velja:
\begin{center}
    \begin{math}
        P(X_B = k) \approx P(X_P = k) = \frac{(np)^k}{k!} e^{-np} 
    \end{math}.
\end{center}

\textbf{8.5.2 Laplaceov Priblizek}

Naj bo 
\begin{center}
    \begin{math}
        X_B \sim B(n, p)\: in\: X_N \sim N(np, \sqrt{np(1-p)})
    \end{math}.
\end{center}
Ce je $np \geq 10$ in $n(1 - p) \geq 10$, potem za $k$ dovolj blizu $np$ velja:
\begin{center}
    \begin{math}
        P(X = k) \approx P(X_N = k) = 
        \frac{
            e^{-(k - np)^2/ (2np(1 - p))}
        }{\sqrt{2 \pi np(1 - p)}}
    \end{math}
\end{center}

\section{\underline{CLI}}

\textbf{9.1 Normalne spremenljivke:} Naj bosta $X \sim N(\mu_1, \sigma_1), Y \sim N(\mu_2, \sigma_2)$ neodvisni.
Potem je 
\begin{center}
    \begin{math}
        X + Y \sim N(\mu_1 + \mu_2, \sqrt{\sigma_1^2 + \sigma_2^2})    
    \end{math}
\end{center}
Posledica: Naj bodo
\begin{center}
    \begin{math}
        X_1 \sim N(\mu_1, \sigma_1), \dots , X_n \sim  N(\mu_n, \sigma_n)
    \end{math}
\end{center}
neodvisne, normalno porazdeljene s.s. Potem velja:
\begin{center}
    \begin{math}
        X_1 + \dots + X_n \sim N (
            \sum_{i=1}^{n} \mu_i, \sqrt{\sum_{i=1}^{n} \sigma_i^2}
        )
    \end{math}.
\end{center}

\textbf{9.2 CLI za vsoto sl. spremenljivk}
Naj bodo $X_1, + \dots +, X_n$ neodvisne in enako porazdeljene
slucajne spremenljivke, kjer velja $E(X_i) = \mu, D(X_i) = \sigma^2 < \infty$. Potem za 
dovolj velik $n$ (dobra aproksimacija : $n \geq 30$)  velja, da je porazdelitev vsote
$S = X_1, + \dots +, X_n$ priblizno normalna.

\begin{center}
    \begin{math}
        S \sim  N(n \mu, \sigma \sqrt{n})
    \end{math}
\end{center}

Pri aproksimaciji \textbf{diskretne} porazdelitvene vsote z normalno porazdelitvijo, uporbljamo
popravek za zveznost.

\begin{center}
    \begin{math}
        P(a \leq S \leq b) \approx P(a - \frac{1}{2} \leq Y \leq b + \frac{1}{2}) =
        F(
            \frac{b + \frac{1}{2} - n \mu}{\sigma \sqrt{n}}
        ) -
        F(
            \frac{b - \frac{1}{2} - n \mu}{\sigma \sqrt{n}}
        )
    \end{math}
\end{center}

\textbf{9.3 Enostavni slucajni vzorec}
Naj bo $X$ s.s. Enostavni slucajni vzorec je slucajni vektor $(X_1, + \dots +, X_n)$,
za katerega velja:
\begin{itemize}
    \item vsi cleni vektorja $X_i$ imajo isto porazdelitev kot spremenljivka $X$ in
    \item cleni $X_i$ so med seboj neodvisni
\end{itemize}

\textbf{9.4 Vzorcno povprecje} normalno porazdeljenega \textbf{vzorca}
Naj bo  $(X_1, + \dots +, X_n)$ enostavni slucajni vzorec, $X_i \sim N(\mu, \sigma)$.
Potem je porazdelitev vzorcnega povprecja $\overline{X} = \frac{1}{n} \sum_{i = 1}^{n} X_i$
tudi normalna:
\begin{center}
    \begin{math}
        \overline{X} \sim N(\mu, \frac{\sigma}{\sqrt{n}})
    \end{math}
\end{center}

Pozor! Pri racunanju disperzije ne pozabi kvadrirati \textbf{$\frac{1}{n}$}.

\textbf{9.5 CLI} za \textbf{vzorcno povprecje}
Naj bo  $(X_1, + \dots +, X_n)$ enostavni slucajni vzorec in
\begin{center}
    \begin{math}
        E(X_i) = \mu, D(X_i) = \sigma^2
    \end{math}
    \begin{math}
        (\mu\; in\; \sigma^2\; morata\; biti\; koncni)
    \end{math}
\end{center}
Za dovolj veliki vzorec $(n \geq 30)$ je porazdelitev vzorcnega povprecja
$\overline{X} = \frac{1}{n} \sum_{i = 1}^{n} X_i$ priblizno normalna
\begin{center}
    \begin{math}
        \overline{X} \sim N(\mu, \frac{\sigma}{\sqrt{n}})
    \end{math}
\end{center}

\textbf{9.6 Racunanje razpona}
Naj bo s.s. $X$ normalno porazdeljena $X \sim N(\mu, \sigma)$
\begin{center}
    \begin{math}
        P(E(X) - a \leq X \leq E(X) + a) = p
    \end{math}\\
    \begin{math}
        I = [E(X) -a, E(X) + a]
    \end{math}
\end{center}


\bigskip
\end{multicols}
\end{document}