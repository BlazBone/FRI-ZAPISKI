\textit{Ponovitev analize}\\

\textbf{appx. Odvodi}
\begin{center}
    \begin{small}
        \begin{enumerate}
            \item \begin{math}
                \frac{1}{x} = -\frac{1}{x^2}
            \end{math}
            \item \begin{math}
                x^n  = nx^{n-1}
            \end{math}
            \item \begin{math}
                \sqrt{x} = \frac{1}{2 \sqrt{x}}
            \end{math}
            \item \begin{math}
                \sqrt[n] x = \frac{1}{n \sqrt[n]{x^{n-1}}}
            \end{math}
            \item \begin{math}
                \sin (a x) =  a  \cos a x
            \end{math}
            \item  \begin{math}
                \cos (a x) = - a \sin (a x)
            \end{math}
            \item \begin{math}
                \tan x = \frac{1}{\cos^2 x} 
            \end{math}
            \item \begin{math}
                e^ax = ae^{ax}
            \end{math}
            \item \begin{math}
                a^x = a^x \ln a
            \end{math}
            \item \begin{math}
                x^x = x^x (1+\ln x)
            \end{math}
            \item \begin{math}
                ln x = \frac{1}{x}
            \end{math}
            \item \begin{math}
                log_a x = \frac{1}{x \ln a}
            \end{math}
            \item \begin{math}
                \arcsin x = \frac{1}{\sqrt {1 - x^2}}
            \end{math}
            \item \begin{math}
                \arccos x = - \frac{1}{\sqrt{1 - x^2}}
            \end{math}
            \item \begin{math}
                \arctan x = \frac{1}{1 + x^2}
            \end{math}
            \item \begin{math}
                \operatorname{arccot}x = -\frac{1}{1 + x^2}
            \end{math}
        \end{enumerate}
    \end{small}
\end{center}
\textbf{appx. Integrali}
\begin{center}
    \begin{small}
        \begin{enumerate}
            \item \begin{math}
                \int x^a\,dx =
                \Bigg\{\begin{tabular}{ccc}
                    $\frac{x^{a+1}}{a+1} + C$  & $a \neq -1$ & \\
                    $ \ln{\left|x\right|} + C$ & $a = -1$ & \\
                  \end{tabular}
            \end{math}
            \item \begin{math}
                \int \ln {x}\,dx = x \ln {x} - x + C
            \end{math}
            \item \begin{math}
                \int \frac {1}{\sqrt{x}}\,dx=2\sqrt{x} + C 
            \end{math}
            \item \begin{math}
                \int e^x\,dx = e^x + C
            \end{math}
            \item \begin{math}
                \int a^x\,dx = \frac{a^x}{\ln{a}} + C
            \end{math}
            \item \begin{math}
                \int \cos({ax}) \, dx = { \frac{sin (ax)}{a} } + C
            \end{math}
            \item \begin{math}
                \int \sin({ax}) \, dx = { \frac{-cos(ax)}{a} } + C
            \end{math}
            \item \begin{math}
                \int \tan{x} \, dx = -\ln{\left| \cos {x} \right|} + C
            \end{math}
            \item \begin{math}
                \int \frac{dx}{\cos^2 x}=\int \sec^2 x \, dx = \tan x + C
            \end{math}
            \item \begin{math}
                \int \frac{dx}{\sin^2 x}=\int \csc^2 x \, dx = -\cot x + C
            \end{math}
            \item \begin{math}
                \int {\frac{1}{\sqrt{1-x^2}}} \, dx = \arcsin {x} + C
            \end{math}
        \end{enumerate}
    \end{small}
\end{center}
    \textbf{Integriranje absolutnih vrednosti} (primer):
    \begin{small}
        Imamo funkcijo $f(x) = |x|$ , ki je zvezna na intervalu $[-1, 1]$
        Ce hocemo to funkcijo in zelimo izracunati njeno 
        \textit{porazdelitveno} funkcijo integrirati locimo 2 primera:    
    \end{small}
    \begin{enumerate}
        \item \begin{math}
            -1 \leq x < 0\\
            F(x) = \int_{-1}^x |t|\,dt = \int_{-1}^x -t\,dt = - \frac{t^2}{2} \rvert_{-1}^{x} = - \frac{1}{2} (x^2 -1)
        \end{math}
        \item \begin{math}
            0 \leq x < 1 \\
            F(x) = \int_{-1}^x |t|\,dt = \int_{-1}^0 -t\,dt + \int_{0}^x t\,dt = - \frac{t^2}{2} \rvert_{0}^{-1} + - \frac{t^2}{2} \rvert_{0}^{x}= \frac{1}{2} (1 + x^2)
        \end{math}
    \end{enumerate}
% other useful formulas
\begin{small}
    \begin{math}
        \sqrt[n] x = (x)^{\frac{1}{n}}
    \end{math}    
\end{small}
