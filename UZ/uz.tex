\documentclass{article}
\usepackage[margin=0.15cm]{geometry}
\usepackage{amsmath}
\usepackage{multicol}
\usepackage{titlesec}
\usepackage{listings}
\usepackage{paralist}
\usepackage{amssymb}

\setcounter{secnumdepth}{4}

\titleformat{\paragraph}
{\normalfont\normalsize\bfseries}{\theparagraph}{1em}{}
\titlespacing*{\paragraph}
{0pt}{3.25ex plus 1ex minus .2ex}{1.5ex plus .2ex}

\begin{document}
	
\begin{center}
	{\small UZ/FRI \par}
\end{center}


\begin{multicols*}{2}
\setlength{\parindent}{0pt}{
	
\section{Pinhole camera}

\begin{compactitem}
	\item Barrier with a hole and a film
	\item Focal length - $f$
	\item Perspective projection equation
	$\frac{f}{-y'}=\frac{-z}{y}$
	\item Apature - size of pinhole
	\item Lense - focusing element
	\item Depth of field - distance and size of small blurring
	\item Field of view - $\varphi=tan^{-1}(\frac{d}{2f})$
	\item Chromatic aberration - different wavelength refraction
	\item Spherical aberration - spherical lenses
	\item Vignetting - edge of camera absorbtion
	\item Radial distortion - lens imperfections
	\item Digitalization - discretization, quantization
\end{compactitem}

\section{Color spaces}

\begin{compactitem}
	\item Additive models \textit{(RGB)} - colors added to black
	\item Subtractive models \textit{(CYMK)} - colors added to white
	\item Linear color space - CIE XYZ
	
	Artificial primaries $X, Y, Z$
	
	$x=\frac{X}{X+Y+Z}$,
	$y=\frac{Y}{X+Y+Z}$,
	$z=\frac{Z}{X+Y+Z}$,
	$x + y + z = 1$
	
	Chromacity is represented using only $[x,y]$
	\item RGB
	\item Nonlinear color space - HSV
	\item Uniform color space - CIE u'v'
\end{compactitem}

\section{Basic image processing}

Basic process

\begin{compactitem}
	\item Localize
	\item Describe
	\item Classify
\end{compactitem}

\subsection{Thresholding}

Transforms an image into a binary mask

\begin{compactitem}
	\item Single(two) threshold approach
	
	$F_T[i,j] = \begin{cases}
		1, \text{if } T_1 \leq F[i,j] (\leq T_2) \\
		0, \text{otherwise}
	\end{cases}$
	
	\item General approach
	
	$F_T[i,j] = \begin{cases}
		1, \text{if } F[i,j] \in Z \\
		0, \text{otherwise}
	\end{cases}$

	\item Global binarization
	
	\textbf{Otsu's method}
	
	Minimizes within class variance
	
	$\sigma^2_{within}(T) = n_1(T)\sigma^2_1(T) + n_2(T)\sigma^2_2(T)$
	$\equiv \text{maximization of}$
	$\sigma^2_{between}(T) = \sigma^2-\sigma^2_{within}(T) = n_1(T)n_2(T)[\mu_1(T)-\mu_2(T)]$
	
	Find $T^* = \text{argmax}_T[\sigma^2_{between}(T)]$
	
	\item Local binarization
	
	Estimate local threshold in neighborhood $W$
	$T_W = \mu_W + k\sigma_W \text{ for } k \in [-1,1]$
	
	\item Shade compensation using polynomials
\end{compactitem}

\subsection{Morphology}

Structuring element
$\begin{cases}
	\text{Fit: all 1's cover 1's in SE} \\
	\text{Hit: at least 1 covers a 1 in SE}
\end{cases}$

\begin{compactitem}
	\item Erosion $g = f \ominus s$
	
	$g(x,y) = \begin{cases}
		1, \text{if s fits f} \\
		0, otherwise
	\end{cases}$

	\item Dilation $g = f \oplus s$
	
	$g(x,y) = \begin{cases}
		1, \text{if s hits f} \\
		0, otherwise
	\end{cases}$

	\item Opening $A \circ B = (A \ominus B) \oplus B$
	- opens gaps, holes
	
	\item Closing $A \bullet B = (A \oplus B) \ominus B$
	- closes gaps, holes
\end{compactitem}

\subsection{Region descriptors}

\textbf{Labeling components}

\begin{compactitem}
	\item 4-8 way connectivity
	\item Connected components
	\begin{lstlisting}
for px in image top to bottom left to right:
 if px == 1:
  if one neighbor top or left:
   label = n_label
  if both neighbors:
   label = n_label if they have same label
   else copy left label and add equivalency
  else:
   label = new label
	\end{lstlisting}

	\item Describe region
	\begin{compactitem}
		\item Area $A$
		\item Perimeter $l$
		\item Compactness $c = l^2(4 \pi A)$
		\item Circularity $l/c$
		\dots
	\end{compactitem}
\end{compactitem}

Color similarity between objects
\begin{compactitem}
	\item Average color
	\item Fit Gaussian distribution
	\item Histograms - $H(c) = \text{number of px with color c}$
	
	Robust to translation, scale, partial occlusion
	
	\item Intensity normalization
	
	$r=\frac{R}{R+G+B}$,
	$g=\frac{G}{R+G+B}$,
	$b=\frac{B}{R+G+B}$
	
	Reduce to 2D ($[r,g]$) since
	$r+g+b=1$	
\end{compactitem}

\textbf{Distances}
\begin{compactitem}
	\item $L_2$ norm (\textit{Euclidean})
	$d(Q,V) = \sqrt{\sum_i(q_i-v_i)^2}$
	
	\item $\chi^2(Q, V) = \sum_i{\frac{(q_i - v_i)^2}{q_i+v_i}}$
	
	\item Hellinger $d_{Hell}(Q, V) = \sqrt{1 - \sum_i{\sqrt{q_i v_i}}}$
\end{compactitem}

\subsection{(Non)linear filters}

Types of noise
\begin{compactitem}
	\item Salt and pepper
	\item Impulse noise
	\item Gaussian noise
\end{compactitem}

\textbf{Convolution}
\begin{compactitem}
	\item Correlation $G = H \otimes F$
	
	$G[i,j] = \sum_{u=-k}^{k}\sum_{v=-l}^{k}H[u,v]F[i+u,j+v]$
	
	\item Convolution $G = H \star F$

	$G[i,j] = \sum_{u=-k}^{k}\sum_{v=-l}^{k}H[u,v]F[i-u,j-v]$
	
	If $H[-u,-v] = H[u,v]$, then $\otimes \equiv \star$
	
	\item Properties
	
	\begin{compactitem}
		\item Linear - $h \star (\alpha_1f_1+\alpha_2f_2) = \alpha_1(h \star f_1) + \alpha_2(h \star f_2)$
		\item Commutative - $f \star g = g \star f$
		\item Associative - $(f \star g) \star h = f \star (g \star h)$
		
		and so also $((f \star b_1) \star b_2) \star b_3 = f \star (b_1 \star b_2 \star b_3)$
		
		\item Derivative -
		$\frac{\partial}{\partial x}(f \star g) = (\frac{\partial}{\partial x}f) \star g = (\frac{\partial}{\partial x}g) \star f$
	\end{compactitem}

	\item Boundry conditions
	\begin{compactitem}
		\item Crop
		\item Bend image around
		\item replicate edges
		\item Mirror image
	\end{compactitem}

	\item Image pyramids
	
	Nyquist theorem - sample the signal by at least $2f$
	- Remove high frequencies before sub-sampling
	
	Gaussian pyramid
	
	$G_i = (G_{i - 1} \star \text{Gaussian}) \downarrow 2$, $G_0 = \text{image}$

\end{compactitem}

\section{Edge detection and image gradients}

\subsection{Image derivatives}

Discrete case (images)

$\frac{\partial f(x,y)}{\partial x} = \frac{f(x+1, y)-f(x,y)}{1}$

Gradient magnitude

$\nabla f = [\frac{\partial f}{\partial x}, \frac{\partial f}{\partial y}]$

Direction - $\theta = tan^{-1}(\frac{\partial f}{\partial x} / \frac{\partial f}{\partial y})$

Magnitude - $\| \nabla f \| = \sqrt{(\frac{\partial f}{\partial x})^2 + (\frac{\partial f}{\partial y})^2}$

Smarter derivative - $\frac{\partial}{\partial x}(I \star G) = I \star (\frac{\partial}{\partial x} G)$

\subsection{Canny edge detector}

Good edge detector

\begin{compactitem}
	\item Detection - minimizes FP and FN
	\item Localization - close to true edge
	\item Specificity - minimize local maxima
\end{compactitem}

\textbf{Process}
\begin{compactenum}
	\item Calculate $I_\partial = I \star \frac{\partial}{\partial x} G$
	\item Calculate $\theta$, $\| \nabla f \|$
	\item Non-maxima suppression
	\item Trace edges by hysteresis thresholding
\end{compactenum}

\subsection{Hough transform}

\textbf{Process}
\begin{compactenum}
	\item For each edge point compute all possible parameters passing through that point
	\item For each set of parameters cast a vote
	\item Select parameter combinations that receive enough votes
\end{compactenum}

Lines - $x \cos \theta - y \sin \theta = d$

Circles - $(x - a)^2 + (y - b)^2 = r^2$

\textbf{Extensions}
\begin{compactitem}
	\item Use gradient direction for $\theta$
	\item Use magnitude for voting weight
	\item Generalized Hough transform
\end{compactitem}

\section{Fitting models}

Transformation of points $x_i' = f(x_i; \mathbf{p})$

\subsection{Least-squares}
Minimizing a continuous error function $\tilde{p} = argmin_\mathbf{p}E(\mathbf{p})$

$\epsilon_i = f(x_i; \mathbf{p}) - y_i$

$E(\mathbf{p}) = \sum_{i=1}^{N}\epsilon_i^2$

Strategy

\begin{compactenum}
	\item Rewrite the cost function $E(\mathbf{p})$ into vector-matrix form
	\item $\frac{\partial E(\mathbf{p})}{\partial \mathbf{p}} = 0; \text{solve for p}$
\end{compactenum}

Derivative of linear and quadratic form

$\frac{\partial \mathbf{A^Tp}}{\partial \mathbf{p}} = \mathbf{A^T}$,
$\mathbf{\frac{\partial p^T A p}{\partial p} = 2 A p}$


\subsection{Normal equations}

Rewrite the error into $\mathbf{A p = b}$

Solve by $\mathbf{p = A^\dagger b = (A^TA)^{-1}A^Tb}$

Weighted least squares

$E(\mathbf{p}) = \sum_{i=1}^{N}w_i\epsilon_i^2$

Solve by $\mathbf{p = (A^TWA)^{-1}A^TWb}$

\subsection{Homogenus systems}

Constrained least squares $Ap = \lambda p$ $\to$ $\mathbf{Ap = 0}$ with $\|p\|^2=1$

Solve by $svd(A)$; $\mathbf{p}$ is eig. vector with smallest eig. value

\subsection{Nonlinear cost function}
\begin{compactitem}
	\item Gradient descend
	\item Newtons method
	\item Levenberg-Marquardt
	\dots
\end{compactitem}

\subsection{RANSAC}

Ransac loop
\begin{compactenum}
	\item Randomly select $s$ correspondences
	\item Fit model parameters
	\item Count projected inliers
	\item Remember the optimal parameters

\end{compactenum}
$e$ - probability of outlier

$s$ - minimal number of correspondences to fit a model

$p$ - probability of drawing all inliers at least once

Number of iterations $N = \frac{\log{(1-p)}}{\log{(1-(1-e)^s)}}$

\section{Keypoints and correspondences}

\subsection{Keypoint detection}

\textbf{Harris corner detector}

$ M = \left[\begin{array}{cc}
	G(\sigma) \star I_x^2 & G(\sigma) \star I_xI_y \\
	G(\sigma) \star I_xI_y & G(\sigma) \star I_y^2
\end{array}\right] = R \left[ \begin{array}{cc}
	\lambda_{max} & 0 \\
	0 & \lambda_{min}
\end{array} \right] R^T$

$det(M) - \alpha trace^2(M) > t$, $0.04 < \alpha < 0.06$

$det(M) = AB - C^2$, $trace(M) = A + B$

$CRF(I) = det(M) - \alpha trace(M)$

Process

\begin{compactenum}
	\item Image derivatives
	\item Squared derivatives
	\item Gaussian filtered squared derivatives
	\item Corner response function
	\item Non-maxima suppression
\end{compactenum}

\textbf{Hessian corner detector}

$Hessian(I) = \left[\begin{array}{cc}
	I_{xx} & I_{xy} \\
	I_{xy} & I_{yy}
\end{array}\right]$

$CRF(I) = det(Hessian(I)) = I_{xx}I_{yy} - I^2_{xy}$

Process

\begin{compactenum}
	\item Image derivatives
	\item Second order derivatives
	\item Corner response function
	\item Non-maxima suppression
\end{compactenum}

\textbf{Laplacian of Gaussian}

Used as scale response function

$LoG = \nabla^2g = \frac{\partial^2g}{\partial x^2} + \frac{\partial^2g}{\partial y^2}$

Process

\begin{compactenum}
	\item Laplacian pyramid
	\item Scale space non-maxima suppression
\end{compactenum}
\textbf{Difference of Gaussian}

$DoG = G(x, y, k \sigma) - G(x, y, \sigma)$

Process

\begin{compactenum}
	\item Gaussian pyramid (faster because of down-sampling)
	\item DoG pyramid from Gaussian
	\item Scale space non-maxima suppression
	\item Remove low contrast points
	\item Remove points detected at edges
\end{compactenum}

\subsection{Local descriptors}

\begin{compactitem}
	\item Vector of region intensities
	\item SIFT
\end{compactitem}

\subsection{SIFT}

\begin{compactenum}
	\item Split region in 4x4 cells
	\item Calculate gradient
	\item 8 bin histogram of gradient weighted by magnitude and region center
	\item Stack histograms and normalize
\end{compactenum}

Rotation invariance

36 bins by angle, rotate gradients using dominant rotation, create descriptor for every orientation with magnitude at least 80\% of maximum.

\textbf{Affine adaptation}

Start with circular window, estimate new window using the covariance matrix of current window. Iterate until convergence.

Rotate $R=U^{-1}$ and scale $S^{-1/2}$ from window ellipse $\Sigma = USU^T$, calculate descriptor using affine adapted region.

\subsection{Correspondences}

\begin{compactitem}
	\item Find most similar descriptor
	\item Keep only symmetric
	\item Keep only distinctive (ratio to second most similar)
\end{compactitem}

\section{Cameras and stereo systems}

Projection: extrinsic $world \to camera$, intrinsic $camera \to image$


\subsection{Pinhole camera model}

$\left[ \begin{array}{c}
	X \\ Y \\ Z \\ 1
\end{array} \right]
\mapsto
\left[ \begin{array}{c}
	fX/Z \\ fY/Z \\ 1
\end{array} \right]
$
$K = \left[ \begin{array}{ccc}
	a_x & s & x_0 \\
	0 & a_y & y_0 \\
	0 & 0 & 1
\end{array} \right]$
$P_0 = K[I|0]$

Principal point $[p_x, p_y]$, focal length $f$, pixels per meter $[m_x, m_y]$, skew $s$

$a_x = f m_x$, $a_y = f m_y$, $x_0 = p_x m_x$, $y_0 = p_y m_y$

$\tilde{X}_{cam} = R(\tilde{X}-\tilde{C}) = 
\left[ \begin{array}{cc}
	R & -R\tilde{C} \\
	0 & 1
\end{array} \right] X$, Camera origin in w.c.s $\tilde{C}$

$P = K[R|t], t=-R\tilde{C}$

Nonlinearity correction using polynomial

$\tilde{x} = x_d + (x_d - c_x)(K_1\rho^2+K_2\rho^4+\dots)$, $\tilde{y} = \dots$

Degrees of freedom

$[p_x, p_y]$: 2, $f$: 1, $[m_x \equiv_{rectangular} m_y]$: 1(2), $s$: 1, $R$: 3, $t$: 3

\subsection{Homography}

$w \mathbf{x' = H x}$

\textbf{Direct linear transformation}

$[a_\times]\equiv \left[ \begin{array}{ccc}
	0 & -a_z & a_y \\
	a_z & 0 & -a_x \\
	-a_y & a_x & 0
\end{array} \right]$;
$\mathbf{x_i' \times H x_i} = 0$; $\mathbf{Ah = 0}$

\textbf{Preconditioning}

$T_{pre} = \left[ \begin{array}{ccc}
	a & 0 & c \\
	0 & b & d \\
	0 & 0 & 1
\end{array} \right]$;
$\tilde{x} = T_{pre}x$;
$\tilde{x}_i' \times \tilde{H} \tilde{x}_i = 0$;
$H = T_{pre}^{'-1} \tilde{H} T_{pre}$

Set $a,b,c,d$ so that mean $\tilde{x}_i$ is 0 and variance is 1

\subsection{Vanishing points}

$v = \left[ \begin{array}{c}
	f X_D / Z_D \\
	f Y_D / Z_D
\end{array} \right]$

\subsection{Calibration}

Estimate $\mathbf{P}$ from a known calibration object.

\begin{compactitem}
	\item Using DLT and preconditioning
	$\mathbf{x_i' \times P x_i} = 0$
	and decompose to  $K[R|t]$
	
	\item Using minimization of error
	
	$\varepsilon_i = \left[ \begin{array}{c}
		\varepsilon_{xi} \\ \varepsilon_{yi}
	\end{array} \right] = (x_i - PX_i)\;$
	$E(P)=\sum_{i=1}^{N}\varepsilon_i^T\varepsilon_i$
	
	\item Multiplane calibration
\end{compactitem}

\subsection{Triangulation}

Using DLT with $[x_{1 \times}]P_1X=0$ and $[x_{2 \times}]P_2X=0$,
then minimize sum of re-projection errors using iterative algorithm
$E(X) = d^2(x_1,P_1X) + d^2(x_2,P_2X)$

\subsection{Epipolar geometry}

Epipolar constraint $X^T([T_\times]RX') = 0$; $x^T E x' = 0$

Essential matrix $E=[T_\times] R$ constrains $x$ and $x'$ in meters

Epipolar line vector

$l' = E^Tx$;
$l = Ex'$

Fundamental matrix constrains $\hat{x}$ and $\hat{x}'$ in pixels

$ F = K^{-T}EK^{'-1}$

Epipolar lines
$\hat{l}' = F^Tx$;
$\hat{l} = Fx'$


Epipole $Fe' = 0$, $F^Te = 0$

\subsection{Simple stereo}

Baseline $b_x$, focal length $f$

3D from disparity

$X = x_L \frac{b_x}{d}$, $Y = y_L \frac{b_x}{d}$, $Z = f \frac{b_x}{d}$

\textbf{Global disparity optimization}

Globally consistent solution

$E_{data}(d_i) = e^{-similarity(d_i)}$

$E(d_i) = E_{data}(d_i) + \lambda E_S(d_i)$

Semi global block matching - apply line based optimization across several directions

\subsection{Structure from motion}

\begin{compactenum}
	\item Find keypoint correspondences
	\item estimate $F$ (weak calibration)
	\item get $E$ from $F$ and $K_1, K_2$
	\item get $[R | t]$
	\item triangluation
\end{compactenum}

\textbf{Normalized 8-point algorithm}

\begin{compactenum}
	\item Precondition with $\mu = 0$ and $\sigma = \sqrt{2}$
	\item Create homogeneous system using correspondences and epipolar constraint
	\item Minimize $\sum_{i=1}^{N}(x_i^TFX_i')$ and solve
	\item Set last $\lambda$ to 0 and reconstruct
	\item Transform to original units $F = T^{'T}\tilde{F}T$

\end{compactenum}

\textbf{Fundamental matrix estimation}

\begin{compactenum}
	\item Find keypoint and correspondences using proximity constraint
	\item filter correspondences by visual similarity
	\item apply RANSAC with 8-point algorithm and epipolar constraint pruning

\end{compactenum}


\subsection{Active stereo}

Project light patterns over the object

Multi band triangulation: Assume smooth surface, project color bands

\section{Feature learning}

\subsection{Natural linear coordinate systems}

\textbf{Principal component analysis}

$\Sigma = \frac{1}{N}\sum_{i=1}^{N}(x_i - \mu)(x_u - \mu)^T = \frac{1}{N}X_dX_d^T, X_d = X - \mu$

Maximize $u^T\Sigma u$ $\rightarrow$ $\Sigma u = \lambda u$, solutions are $U$ = eig. vectors of $\Sigma$.

Project data to PCA c.s. $y_i = U^T(x_i-\mu)$

Project data from PCA c.s. $x_i = U y_i + \mu$

\textbf{Dual PCA}

If sample dimension $M >$  number of samples $N$

$\Sigma' = \frac{1}{N}X_d^TX_d$

$u_i = \frac{X_d u_i'}{\sqrt{N \lambda_i'}}$

\textbf{Classification by subspace recognition}

If window contains a trained subspace the reconstruction will work well. $\|\tilde{x}_i-x_i\|^2 < \theta$

\textbf{Linear discriminant analysis}

$S_W = \sum_{i=1}^{c}\sum_{j}(x_j^{(i)} - \mu_i)(x_j^{(i)} - \mu_i)^T$

$S_B = \sum_{i=1}^{c}N_i(\mu_i - \mu)(\mu_i - \mu)^T$

Maximize $J(w) = \frac{w^TS_bw}{w^TS_ww}$ $\rightarrow$ $S_w^{-1}S_bw=\lambda w$, solutions are W = first $c - 1$ eig. vectors of $S_w^{-1}S_bw$.

\subsection{Nonlinear hand-crafted transforms}

\textbf{Histogram of gradients}

\begin{compactenum}
	\item Calculate gradient
	\item Calculate HOG in 8x8 blocks and normalize, weighted by magnitude
	\item Train a classifier using support vector machine
\end{compactenum}

\subsection{Feature selection}

\textbf{Viola-Jones face detection}

Boosting (Adaboost)
\begin{compactitem}
	\item Strong classifier from many weak classifiers
	
	$h(x) = sign(\sum_{t=1}^{T} \alpha_t h_t(x) )$, classifier weight $\alpha_t$, weak classifier $h_t(x)$
	
	\item Weak classifiers using sum of region intensities with integral images $\Sigma (R) = \int_\ulcorner + \int_\lrcorner - \int_\llcorner - \int_\urcorner$
	
	\item Using cascade of classifiers to reject obvious windows
\end{compactitem}

\textbf{Region proposals for selective search}

Can be used by slow classifiers, hierarchical segmentation

\subsection{End-to-end feature \& classifier learning}

\textbf{Convolutional neural networks}

\begin{compactitem}
	\item[] \textbf{Feature extraction}
	\item Convolutional layers
	\item Nonlinearity (RELU)
	\item Pooling layers
	\item[] \textbf{Classifier}
	\item Multi-layer perceptron
\end{compactitem}

\textbf{Region based CNN evolution}
\begin{compactitem}
	\item Slow R-CNN - processes every region proposal through the whole CNN to classify
	\item Fast R-CNN - process whole image to extract features and join with region proposals to classify
	\item Faster R-CNN - generate region proposals using extracted features network, multi-scale feature extraction	
	\item Mask R-CNN - Use box regression and additional MLP to create the segmentation mask
\end{compactitem}


\section{Keypoint based recognition}

\subsection{Bag of words models}

\begin{compactenum}
	\item Feature detection and representation
	
	Use SIFT and affine adaptation, collect all descriptors
	
	\item Dictionary construction
	
	Cluster descriptors into clusters (K-means) - cluster center is a word
	
	\item Image representation
	
	Detect words in training images and build word histograms (BOWs)
	
	\item Build a classifier
	
	Use histograms to make a classifier
	
	\item Recognition
	
	Extract BOWs and apply them to the classifier
\end{compactenum}

\subsection{Detection by RANSAC and GHT}

\textbf{Detection by RANSAC}
\begin{compactenum}
	\item Represent model using affine deformation invariant parts
	\item Detect parts in image
	\item Use RANSAC with parts and try to fit a good model
\end{compactenum}

\textbf{Generalized Hough transform}
\begin{compactenum}
	\item Index descriptors
	\item Apply GHT to obtain detections, each feature casts a vote into the Hough space
	\item Refine detection using affine transform
\end{compactenum}

}
\end{multicols*}
\end{document}
